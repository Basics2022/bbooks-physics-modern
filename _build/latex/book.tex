%% Generated by Sphinx.
\def\sphinxdocclass{jupyterBook}
\documentclass[letterpaper,10pt,italian]{jupyterBook}
\ifdefined\pdfpxdimen
   \let\sphinxpxdimen\pdfpxdimen\else\newdimen\sphinxpxdimen
\fi \sphinxpxdimen=.75bp\relax
\ifdefined\pdfimageresolution
    \pdfimageresolution= \numexpr \dimexpr1in\relax/\sphinxpxdimen\relax
\fi
%% let collapsible pdf bookmarks panel have high depth per default
\PassOptionsToPackage{bookmarksdepth=5}{hyperref}
%% turn off hyperref patch of \index as sphinx.xdy xindy module takes care of
%% suitable \hyperpage mark-up, working around hyperref-xindy incompatibility
\PassOptionsToPackage{hyperindex=false}{hyperref}
%% memoir class requires extra handling
\makeatletter\@ifclassloaded{memoir}
{\ifdefined\memhyperindexfalse\memhyperindexfalse\fi}{}\makeatother

\PassOptionsToPackage{warn}{textcomp}

\catcode`^^^^00a0\active\protected\def^^^^00a0{\leavevmode\nobreak\ }
\usepackage{cmap}
\usepackage{fontspec}
\defaultfontfeatures[\rmfamily,\sffamily,\ttfamily]{}
\usepackage{amsmath,amssymb,amstext}
\usepackage{polyglossia}
\setmainlanguage{italian}



\setmainfont{FreeSerif}[
  Extension      = .otf,
  UprightFont    = *,
  ItalicFont     = *Italic,
  BoldFont       = *Bold,
  BoldItalicFont = *BoldItalic
]
\setsansfont{FreeSans}[
  Extension      = .otf,
  UprightFont    = *,
  ItalicFont     = *Oblique,
  BoldFont       = *Bold,
  BoldItalicFont = *BoldOblique,
]
\setmonofont{FreeMono}[
  Extension      = .otf,
  UprightFont    = *,
  ItalicFont     = *Oblique,
  BoldFont       = *Bold,
  BoldItalicFont = *BoldOblique,
]



\usepackage[Sonny]{fncychap}
\ChNameVar{\Large\normalfont\sffamily}
\ChTitleVar{\Large\normalfont\sffamily}
\usepackage[,numfigreset=1,mathnumfig]{sphinx}

\fvset{fontsize=\small}
\usepackage{geometry}


% Include hyperref last.
\usepackage{hyperref}
% Fix anchor placement for figures with captions.
\usepackage{hypcap}% it must be loaded after hyperref.
% Set up styles of URL: it should be placed after hyperref.
\urlstyle{same}


\usepackage{sphinxmessages}



        % Start of preamble defined in sphinx-jupyterbook-latex %
         \usepackage[Latin,Greek]{ucharclasses}
        \usepackage{unicode-math}
        % fixing title of the toc
        \addto\captionsenglish{\renewcommand{\contentsname}{Contents}}
        \hypersetup{
            pdfencoding=auto,
            psdextra
        }
        % End of preamble defined in sphinx-jupyterbook-latex %
        

\title{basics book template}
\date{14 dic 2024}
\release{}
\author{basics}
\newcommand{\sphinxlogo}{\vbox{}}
\renewcommand{\releasename}{}
\makeindex
\begin{document}

\pagestyle{empty}
\sphinxmaketitle
\pagestyle{plain}
\sphinxtableofcontents
\pagestyle{normal}
\phantomsection\label{\detokenize{intro::doc}}


\sphinxAtStartPar
If you want ot start a new basics\sphinxhyphen{}book, it could be a good idea to start from this template.

\sphinxAtStartPar
Please check out the Github repo of the project, \sphinxhref{https://github.com/Basics2022}{basics\sphinxhyphen{}book project}, and the \sphinxhref{https://basics2022.github.io/bbooks}{landing page of the project}.
\begin{itemize}
\item {} 
\sphinxAtStartPar
{\hyperref[\detokenize{ch/relativity-special::doc}]{\sphinxcrossref{Special Relativity}}}

\item {} 
\sphinxAtStartPar
{\hyperref[\detokenize{ch/relativity-general::doc}]{\sphinxcrossref{General Relativity}}}

\item {} 
\sphinxAtStartPar
{\hyperref[\detokenize{ch/statistical-mechanics::doc}]{\sphinxcrossref{Statistical Physics}}}

\item {} 
\sphinxAtStartPar
{\hyperref[\detokenize{ch/quantum-mechanics::doc}]{\sphinxcrossref{Quantum Mechanics}}}

\end{itemize}

\sphinxstepscope


\chapter{Special Relativity}
\label{\detokenize{ch/relativity-special:special-relativity}}\label{\detokenize{ch/relativity-special:relativity-special}}\label{\detokenize{ch/relativity-special::doc}}
\sphinxstepscope


\chapter{General Relativity}
\label{\detokenize{ch/relativity-general:general-relativity}}\label{\detokenize{ch/relativity-general:relativity-general}}\label{\detokenize{ch/relativity-general::doc}}
\sphinxstepscope


\chapter{Statistical Physics}
\label{\detokenize{ch/statistical-mechanics:statistical-physics}}\label{\detokenize{ch/statistical-mechanics:statistical-mechanics}}\label{\detokenize{ch/statistical-mechanics::doc}}
\sphinxstepscope


\chapter{Quantum Mechanics}
\label{\detokenize{ch/quantum-mechanics:quantum-mechanics}}\label{\detokenize{ch/quantum-mechanics:id1}}\label{\detokenize{ch/quantum-mechanics::doc}}

\section{Mathematical tools for quantum mechanics}
\label{\detokenize{ch/quantum-mechanics:mathematical-tools-for-quantum-mechanics}}\label{ch/quantum-mechanics:definition-0}
\begin{sphinxadmonition}{note}{Definition 1 (Operator)}


\end{sphinxadmonition}
\label{ch/quantum-mechanics:definition-1}
\begin{sphinxadmonition}{note}{Definition 2 (Adjoint operator)}



\sphinxAtStartPar
Given an operator \(\hat{A}: U \rightarrow V\), its self\sphinxhyphen{}adjoint \(\hat{A}^*: V \rightarrow U\) is the operator s.t.
\begin{equation*}
\begin{split}(\mathbf{v}, \ \hat{A} \mathbf{u})_{V} = (\mathbf{u}, \hat{A}^* \mathbf{v} )_{U}\end{split}
\end{equation*}
\sphinxAtStartPar
holds for \(\forall \mathbf{u} \in U, \ \mathbf{v} \in V\).
\end{sphinxadmonition}
\label{ch/quantum-mechanics:definition-2}
\begin{sphinxadmonition}{note}{Definition 3 (Hermitian (self\sphinxhyphen{}adjoint) operator)}



\sphinxAtStartPar
If \(\hat{A}: U \rightarrow U\), it is a self\sphinxhyphen{}adjoint operator if \(\hat{A}^* = \hat{A}\).
\end{sphinxadmonition}

\sphinxAtStartPar
Self\sphinxhyphen{}adjoint operators have real eigenvalues, and orthogonal eigenvectors (at least those associated to different eigenvalues; those associated with the same eigenvalues can be used to build an orthogonal set of vectors with orthogonalization process).


\section{Postulates of Quantum Mechanics}
\label{\detokenize{ch/quantum-mechanics:postulates-of-quantum-mechanics}}\begin{itemize}
\item {} 
\sphinxAtStartPar
…

\item {} 
\sphinxAtStartPar
Canonical Commutation Relation (CCR) \sphinxstyleemphasis{and Canonical Anti\sphinxhyphen{}Commutation Relation…}

\end{itemize}


\section{Non\sphinxhyphen{}relativistic Mechanics}
\label{\detokenize{ch/quantum-mechanics:non-relativistic-mechanics}}

\subsection{Statistical Interpretation}
\label{\detokenize{ch/quantum-mechanics:statistical-interpretation}}

\subsubsection{Wave function}
\label{\detokenize{ch/quantum-mechanics:wave-function}}
\sphinxAtStartPar
The state of a system is described by a wave function \(|\Psi\rangle\)

\sphinxAtStartPar
\sphinxstylestrong{todo}
\begin{itemize}
\item {} 
\sphinxAtStartPar
properties: domain, image,…

\item {} 
\sphinxAtStartPar
unitary \(1 = \langle \Psi | \Psi \rangle = \left| \Psi \right|^2\), for statistical interpretation of \(\left| \Psi \right|^2\) as a density probability function

\end{itemize}


\subsubsection{Operators and Observables}
\label{\detokenize{ch/quantum-mechanics:operators-and-observables}}
\sphinxAtStartPar
Physical \sphinxstylestrong{observable} quantities are represented by {\hyperref[\detokenize{ch/quantum-mechanics:quantum-mechanics-math-operators-self-adjoint}]{\sphinxcrossref{\DUrole{std,std-ref}{Hermitian operators}}}}. Possible outcomes of measurement are the eigenvalues of the operator

\sphinxAtStartPar
Given \(\hat{A}\) and the set of its eigenvectors \(\{ |A_i \rangle \}_i\) (\sphinxstylestrong{todo} \sphinxstyleemphasis{continuous or discrete spectrum…, need to treat this difference quite in details}), with associated eigenvalues \(\{ a_i \}_i\)
\begin{equation*}
\begin{split}\hat{A} |A_i \rangle = a_i |A_i\rangle\end{split}
\end{equation*}\begin{equation*}
\begin{split}| \Psi \rangle = | A_i \rangle \langle A_i | \Psi \rangle =  | A_i \rangle \Psi_i^A \end{split}
\end{equation*}\begin{equation*}
\begin{split}\langle A_j | \Psi \rangle = \langle A_j | A_i \rangle \langle A_i | \Psi \rangle = \Psi_j^A \end{split}
\end{equation*}
\sphinxAtStartPar
and thus
\begin{equation*}
\begin{split}\begin{aligned}
  \Psi_j^A    & = \langle A_j  | \Psi \rangle \\
  \Psi_j^{A*} & = \langle \Psi | A_j \rangle \\
\end{aligned}\end{split}
\end{equation*}\begin{itemize}
\item {} 
\sphinxAtStartPar
identity operator \(\sum_i | A_i \rangle \langle A_i | = \mathbb{I}\), since
\begin{equation*}
\begin{split}\sum_i | A_i \rangle \langle A_i | \Psi \rangle = \sum_i | A_i \rangle \langle A_i | \Psi_j^A A_j \rangle =
  \sum_i | A_i \rangle \delta_{ij} \Psi_j^A = \sum_i | A_i \rangle  \Psi_i^A  = | \Psi \rangle\end{split}
\end{equation*}
\item {} 
\sphinxAtStartPar
Normalization:
\begin{equation*}
\begin{split}1 =\langle \Psi | \Psi \rangle = \Psi_j^{A*} \underbrace{\langle A_j | A_i \rangle}_{\delta_{ij}} \Psi_i^{A} = \sum_i \left| \Psi_i^A \right|^2\end{split}
\end{equation*}
\sphinxAtStartPar
with \(|\Psi_i^A|^2\) that can be interpreted as the probability of finding the system in state \(|\Psi_i^a\rangle\)

\item {} 
\sphinxAtStartPar
Expected value of the physical quantity in the a state \(|\Psi\rangle\), with possible values \(a_i\) with probability \(|\Psi_i^A|^2\)
\begin{equation*}
\begin{split}\begin{aligned}
    \bar{A}_{\Psi} & = \sum_i a_i |\Psi_i^A|^2 = \\
            & = \sum_i a_i \Psi_i^{A*} \Psi_i^A =  \\ 
            & = \sum_i a_i \langle \Psi | A_i \rangle \langle A_i | \Psi \rangle = \\
            & = \langle \Psi | \left( \sum_i a_i | A_i \rangle \langle A_i | \right) | \Psi \rangle = \\
            & = \langle \Psi | \hat{A} | \Psi \rangle = \\
  \end{aligned}\end{split}
\end{equation*}
\end{itemize}




\subsubsection{Space Representation}
\label{\detokenize{ch/quantum-mechanics:space-representation}}

\subsubsection{Momentum Representation}
\label{\detokenize{ch/quantum-mechanics:momentum-representation}}

\subsection{Schrodinger Equation}
\label{\detokenize{ch/quantum-mechanics:schrodinger-equation}}\begin{equation*}
\begin{split}i \hbar \dfrac{d}{dt} | \Psi \rangle = \hat{H} | \Psi \rangle \end{split}
\end{equation*}
\sphinxAtStartPar
being \(\hat{H}\) the Hamiltonian operator and \(|\Psi\rangle\) the wave function, as a function of time \(t\) as an independent variable.


\subsubsection{Stationary States}
\label{\detokenize{ch/quantum-mechanics:stationary-states}}
\sphinxAtStartPar
Eigenspace of the Hamiltonian operator
\begin{equation*}
\begin{split}\hat{H} |\Psi_k \rangle = E_i |\Psi_k \rangle \ ,\end{split}
\end{equation*}
\sphinxAtStartPar
with \(E_k\) possible values of energy measurements. \sphinxstyleemphasis{If no eigenstates with the same eigenvalue exists, then…otherwise…}
\sphinxstyleemphasis{Without external influence} \sphinxstylestrong{todo} \sphinxstyleemphasis{be more detailed!}, energy values and eigenstates of the systems are constant in time.

\sphinxAtStartPar
Thus, exapnding the state of the system \(|\Psi\rangle\) over the stationary states gives \(|\Psi_k\rangle\), \(|\Psi\rangle = |\Psi_k\rangle c_k(t)\), and inserting in Schrodinger equation
\begin{equation*}
\begin{split}i \hbar \dot{c}_k |\Psi_k\rangle = c_k  E_k |\Psi_k\rangle\end{split}
\end{equation*}
\sphinxAtStartPar
and exploiting orthogonality of eigenstates, a diagonal system for the amplitudes of stationary states ariese,
\begin{equation*}
\begin{split}i \hbar \dot{c}_k = c_k E_k \ .\end{split}
\end{equation*}
\sphinxAtStartPar
whose solution reads
\begin{equation*}
\begin{split}c_k(t) = c_{k,0} \exp \left[- i \frac{E_k}{\hbar} t \right]\end{split}
\end{equation*}
\sphinxAtStartPar
Thus the state of the system evolves like a superposition of monochromatic waves with frequencies \(\omega_k = \frac{E_k}{\hbar}\),
\begin{equation*}
\begin{split}| \Psi \rangle = | \Psi_k \rangle c_k(t) = | \Psi_k \rangle c_{k,0}\exp\left[-i \frac{E_k}{\hbar} t \right] \ .\end{split}
\end{equation*}

\subsubsection{Representations}
\label{\detokenize{ch/quantum-mechanics:representations}}

\paragraph{Schrodinger}
\label{\detokenize{ch/quantum-mechanics:schrodinger}}

\paragraph{Heisenberg}
\label{\detokenize{ch/quantum-mechanics:heisenberg}}

\paragraph{…}
\label{\detokenize{ch/quantum-mechanics:id2}}

\subsection{Matrix Mechanics}
\label{\detokenize{ch/quantum-mechanics:matrix-mechanics}}

\subsubsection{Attualization of 1925 papers}
\label{\detokenize{ch/quantum-mechanics:attualization-of-1925-papers}}\begin{equation*}
\begin{split}|\Psi\rangle = | \Psi_k \rangle c_{k,0}\exp\left[-i \frac{E_k}{\hbar} t \right]\end{split}
\end{equation*}\begin{equation*}
\begin{split}\langle \Psi | \hat{\mathbf{X}} | \Psi \rangle\end{split}
\end{equation*}\begin{equation*}
\begin{split}\langle \Psi | \Psi_m \rangle \langle \Psi_m | \hat{\mathbf{X}} | \Psi_n \rangle \langle \Psi_n | \Psi \rangle\end{split}
\end{equation*}\begin{equation*}
\begin{split}X_{mn} = \langle \Psi_m | \hat{\mathbf{X}} | \Psi_n \rangle \end{split}
\end{equation*}\begin{equation*}
\begin{split}\langle \Psi | \hat{\mathbf{P}} | \Psi \rangle\end{split}
\end{equation*}

\section{Many\sphinxhyphen{}body problem}
\label{\detokenize{ch/quantum-mechanics:many-body-problem}}
\sphinxAtStartPar
Wave function with symmetries: Fermions and Bosons






\renewcommand{\indexname}{Proof Index}
\begin{sphinxtheindex}
\let\bigletter\sphinxstyleindexlettergroup
\bigletter{definition\sphinxhyphen{}0}
\item\relax\sphinxstyleindexentry{definition\sphinxhyphen{}0}\sphinxstyleindexextra{ch/quantum\sphinxhyphen{}mechanics}\sphinxstyleindexpageref{ch/quantum-mechanics:\detokenize{definition-0}}
\indexspace
\bigletter{definition\sphinxhyphen{}1}
\item\relax\sphinxstyleindexentry{definition\sphinxhyphen{}1}\sphinxstyleindexextra{ch/quantum\sphinxhyphen{}mechanics}\sphinxstyleindexpageref{ch/quantum-mechanics:\detokenize{definition-1}}
\indexspace
\bigletter{definition\sphinxhyphen{}2}
\item\relax\sphinxstyleindexentry{definition\sphinxhyphen{}2}\sphinxstyleindexextra{ch/quantum\sphinxhyphen{}mechanics}\sphinxstyleindexpageref{ch/quantum-mechanics:\detokenize{definition-2}}
\end{sphinxtheindex}

\renewcommand{\indexname}{Indice}
\printindex
\end{document}