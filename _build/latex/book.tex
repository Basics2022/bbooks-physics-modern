%% Generated by Sphinx.
\def\sphinxdocclass{jupyterBook}
\documentclass[letterpaper,10pt,english]{jupyterBook}
\ifdefined\pdfpxdimen
   \let\sphinxpxdimen\pdfpxdimen\else\newdimen\sphinxpxdimen
\fi \sphinxpxdimen=.75bp\relax
\ifdefined\pdfimageresolution
    \pdfimageresolution= \numexpr \dimexpr1in\relax/\sphinxpxdimen\relax
\fi
%% let collapsible pdf bookmarks panel have high depth per default
\PassOptionsToPackage{bookmarksdepth=5}{hyperref}
%% turn off hyperref patch of \index as sphinx.xdy xindy module takes care of
%% suitable \hyperpage mark-up, working around hyperref-xindy incompatibility
\PassOptionsToPackage{hyperindex=false}{hyperref}
%% memoir class requires extra handling
\makeatletter\@ifclassloaded{memoir}
{\ifdefined\memhyperindexfalse\memhyperindexfalse\fi}{}\makeatother

\PassOptionsToPackage{warn}{textcomp}

\catcode`^^^^00a0\active\protected\def^^^^00a0{\leavevmode\nobreak\ }
\usepackage{cmap}
\usepackage{fontspec}
\defaultfontfeatures[\rmfamily,\sffamily,\ttfamily]{}
\usepackage{amsmath,amssymb,amstext}
\usepackage{polyglossia}
\setmainlanguage{english}



\setmainfont{FreeSerif}[
  Extension      = .otf,
  UprightFont    = *,
  ItalicFont     = *Italic,
  BoldFont       = *Bold,
  BoldItalicFont = *BoldItalic
]
\setsansfont{FreeSans}[
  Extension      = .otf,
  UprightFont    = *,
  ItalicFont     = *Oblique,
  BoldFont       = *Bold,
  BoldItalicFont = *BoldOblique,
]
\setmonofont{FreeMono}[
  Extension      = .otf,
  UprightFont    = *,
  ItalicFont     = *Oblique,
  BoldFont       = *Bold,
  BoldItalicFont = *BoldOblique,
]



\usepackage[Bjarne]{fncychap}
\usepackage[,numfigreset=1,mathnumfig]{sphinx}

\fvset{fontsize=\small}
\usepackage{geometry}


% Include hyperref last.
\usepackage{hyperref}
% Fix anchor placement for figures with captions.
\usepackage{hypcap}% it must be loaded after hyperref.
% Set up styles of URL: it should be placed after hyperref.
\urlstyle{same}

\addto\captionsenglish{\renewcommand{\contentsname}{Special Relativity}}

\usepackage{sphinxmessages}



        % Start of preamble defined in sphinx-jupyterbook-latex %
         \usepackage[Latin,Greek]{ucharclasses}
        \usepackage{unicode-math}
        % fixing title of the toc
        \addto\captionsenglish{\renewcommand{\contentsname}{Contents}}
        \hypersetup{
            pdfencoding=auto,
            psdextra
        }
        % End of preamble defined in sphinx-jupyterbook-latex %
        

\title{Modern Physics}
\date{Apr 11, 2025}
\release{}
\author{basics}
\newcommand{\sphinxlogo}{\vbox{}}
\renewcommand{\releasename}{}
\makeindex
\begin{document}

\pagestyle{empty}
\sphinxmaketitle
\pagestyle{plain}
\sphinxtableofcontents
\pagestyle{normal}
\phantomsection\label{\detokenize{intro::doc}}


\sphinxAtStartPar
This material is part of the \sphinxhref{https://basics2022.github.io/bbooks}{\sphinxstylestrong{basics\sphinxhyphen{}books project}}. It is also available as a \DUrole{xref,download,myst}{.pdf document}.

\sphinxAtStartPar
Please check out the Github repo of the project, \sphinxhref{https://github.com/Basics2022}{basics\sphinxhyphen{}book project}.
\begin{itemize}
\item {} 
\sphinxAtStartPar
Special Relativity

\begin{itemize}
\item {} 
\sphinxAtStartPar
{\hyperref[\detokenize{ch/relativity-special/intro::doc}]{\sphinxcrossref{Special Relativity}}}

\item {} 
\sphinxAtStartPar
{\hyperref[\detokenize{ch/relativity-special/notes::doc}]{\sphinxcrossref{Special Relativity \sphinxhyphen{} Notes}}}

\end{itemize}
\end{itemize}
\begin{itemize}
\item {} 
\sphinxAtStartPar
General Relativity

\begin{itemize}
\item {} 
\sphinxAtStartPar
{\hyperref[\detokenize{ch/relativity-general/intro::doc}]{\sphinxcrossref{General Relativity}}}

\item {} 
\sphinxAtStartPar
{\hyperref[\detokenize{ch/relativity-general/notes::doc}]{\sphinxcrossref{General Relativity \sphinxhyphen{} Notes}}}

\end{itemize}
\end{itemize}
\begin{itemize}
\item {} 
\sphinxAtStartPar
Statistical Mechanics

\begin{itemize}
\item {} 
\sphinxAtStartPar
{\hyperref[\detokenize{ch/statistical-mechanics/intro::doc}]{\sphinxcrossref{Statistical Physics}}}

\item {} 
\sphinxAtStartPar
{\hyperref[\detokenize{ch/statistical-mechanics/notes::doc}]{\sphinxcrossref{Statistical Physics \sphinxhyphen{} Notes}}}

\item {} 
\sphinxAtStartPar
{\hyperref[\detokenize{ch/statistical-mechanics/statistics::doc}]{\sphinxcrossref{Statistical Physics \sphinxhyphen{} Statistics Miscellanea}}}

\end{itemize}
\end{itemize}
\begin{itemize}
\item {} 
\sphinxAtStartPar
Quantum Mechanics

\begin{itemize}
\item {} 
\sphinxAtStartPar
{\hyperref[\detokenize{ch/quantum-mechanics/intro::doc}]{\sphinxcrossref{Quantum Mechanics}}}

\item {} 
\sphinxAtStartPar
{\hyperref[\detokenize{ch/quantum-mechanics/notes::doc}]{\sphinxcrossref{Quantum Mechanics \sphinxhyphen{} Notes}}}

\end{itemize}
\end{itemize}

\sphinxstepscope


\part{Special Relativity}

\sphinxstepscope


\chapter{Special Relativity}
\label{\detokenize{ch/relativity-special/intro:special-relativity}}\label{\detokenize{ch/relativity-special/intro:relativity-special}}\label{\detokenize{ch/relativity-special/intro::doc}}\begin{itemize}
\item {} 
\sphinxAtStartPar
Electromagnetism and the need for new relativity

\item {} 
\sphinxAtStartPar
Space\sphinxhyphen{}time, Lorentz transformations,…

\item {} 
\sphinxAtStartPar
Mechanics: kinematics, dynamics,…

\item {} 
\sphinxAtStartPar
Electromagnetism: Maxwell’s equations, potentials, Lorentz force, energy balance

\end{itemize}

\sphinxstepscope


\chapter{Special Relativity \sphinxhyphen{} Notes}
\label{\detokenize{ch/relativity-special/notes:special-relativity-notes}}\label{\detokenize{ch/relativity-special/notes:relativity-special-notes}}\label{\detokenize{ch/relativity-special/notes::doc}}
\sphinxAtStartPar
An event is determined by spatio\sphinxhyphen{}temporal information together, \(t, \vec{r}\). Absolute nature of physics needs vector algebra and calculus formalism
\begin{equation*}
\begin{split}\mathbf{X} = c  \,t  \,\mathbf{e}_0 + \vec{r} =  c \, t \, \mathbf{e}_0 + x^1 \mathbf{e}_1 + x^2 \mathbf{e}_2 + x^3 \mathbf{e}_3 = X^{\alpha} \mathbf{E}_{\alpha} \ ,\end{split}
\end{equation*}
\sphinxAtStartPar
having used Cartesian coordiantes for the space coordinate.

\sphinxAtStartPar
Minkowski metric reads
\begin{equation*}
\begin{split}g_{\alpha \beta} = \mathbf{E}_{\alpha} \cdot \mathbf{E}_{\beta} = \text{diag}\{-1, 1, 1, 1\}\end{split}
\end{equation*}
\sphinxAtStartPar
The reciprocal basis reads \(\mathbf{E}_{\alpha} \cdot \mathbf{E}^{\beta} = \delta_{\alpha}^{\beta}\), \(\mathbf{E}_{\alpha} = g_{\alpha \beta} \mathbf{E}^{\beta}\), s.t. the elementary interval between two events can be written as
\begin{equation*}
\begin{split}d \mathbf{X} = d X^{\alpha} \, \mathbf{E}_{\alpha} = \underbrace{d X^{\alpha} \, g_{\alpha \beta}}_{= dX_{\beta}} \, \mathbf{E}^{\beta} = d X_{\beta} \, \mathbf{E}^{\beta} \ ,\end{split}
\end{equation*}
\sphinxAtStartPar
having used Cartesian coordinates,
\begin{equation*}
\begin{split}
  \begin{aligned}  & X^0 = c t \\  & X_0 = c t \end{aligned} \qquad
  \begin{aligned}  & X^1 =   x \\  & X_1 =  -x \end{aligned} \qquad
  \begin{aligned}  & X^2 =   y \\  & X_2 =  -y \end{aligned} \qquad
  \begin{aligned}  & X^3 =   z \\  & X_3 =  -z \end{aligned}
\end{split}
\end{equation*}
\sphinxAtStartPar
Its “length”, or better pseudo\sphinxhyphen{}norm with Minkowski metric, is invariant and reads
\begin{equation*}
\begin{split}d s^2 = d \mathbf{X} \cdot d \mathbf{X} = \left( dX_{\alpha} \mathbf{E}^{\alpha} \right) \cdot \left( dX^{\beta} \mathbf{E}_{\beta} \right) = c^2 \, d t^2 - (dx^1)^2 - (dx^2)^2 - (dx^3)^2 = c^2 \, dt^2 - |d \vec{r}|^2 \end{split}
\end{equation*}
\begin{sphinxadmonition}{note}{Note:}
\sphinxAtStartPar
\(ds\) is invariant
\sphinxstylestrong{todo} prove it. And/or add a section about the role of invariance.
\end{sphinxadmonition}

\sphinxAtStartPar
For a co\sphinxhyphen{}moving observer, \(d \vec{r}' = \vec{0}\), and \(t'\) is commonly indicated with \(\tau\), and its differential is invariant itself, being the product of a constant (\(c\) is a universal constant in special relativity) and an invariant quantity.
\begin{equation*}
\begin{split}d s^2 = c^2 dt'^2 - |d \vec{r}'|^2 = c^2 d \tau^2 \ .\end{split}
\end{equation*}
\sphinxAtStartPar
Given the invariant nature of \(d s\),
\begin{equation*}
\begin{split}d s^2 = c^2 \, d \tau^2 = c^2 \, dt^2 - |d \vec{r}|^2 = c^2 \, dt^2 \left[ 1 - \frac{1}{c^2}\frac{|d\vec{r}|^2}{dt^2} \right] = c^2 \, dt^2 \left[ 1 - \frac{|\vec{v}|^2}{c^2} \right]\end{split}
\end{equation*}
\sphinxAtStartPar
and thus
\begin{equation*}
\begin{split}d s = c \, d \tau = \gamma^{-1}(v/c) \, c \, dt \ ,\end{split}
\end{equation*}
\sphinxAtStartPar
with \(\gamma(w) = \frac{1}{\sqrt{1 - w^2}}\).

\sphinxAtStartPar
\sphinxstylestrong{4\sphinxhyphen{}Velocity} Given the parametric representation of an event in space\sphinxhyphen{}time as a function of its proper time, \(\mathbf{X}(\tau)\) or coordinate \(s\), \(\mathbf{X}(s)\) the derivative w.r.t. this parameter is defined as the 4\sphinxhyphen{}velocity of the event in space time. Using Cartesian coordinates inducing constant and uniform basis \(\mathbf{E}_{\alpha}\), as a function of the observer time \(t\), \(c t\), \(x^i(t)\), and the transformation of coordinates \(t(\tau)\), with differential \(d t = \frac{1}{\gamma} \, d \tau\)
\begin{equation*}
\begin{split}\mathbf{U}(\tau) := \mathbf{X}'(\tau) = \dfrac{d}{d \tau} \left( X^{\alpha}(\tau) \mathbf{E}_{\alpha} \right) = \dfrac{d t}{d \tau} (c t \mathbf{E}_0 + x^i(t) \mathbf{E}_i) = \gamma(v/c) \left( c \mathbf{E}_0 + \dot{x}^i(t) \mathbf{E}_i \right) = \gamma(v/c) \left( c \mathbf{E}_0 + \vec{v} \right)\end{split}
\end{equation*}
\sphinxAtStartPar
or
\begin{equation*}
\begin{split}\mathbf{U}(s) := \mathbf{X}'(s) = \dfrac{d t}{ ds } \dfrac{d}{dt} \mathbf{X}(t) = \dots = \gamma(v/c) \left( \mathbf{E}_0 + \frac{\vec{v}}{c} \right) \ .\end{split}
\end{equation*}
\begin{sphinxadmonition}{note}{Note:}
\sphinxAtStartPar
Using \(s\) as the parameter, \(\mathbf{U}\) is non\sphinxhyphen{}dimensional, and has pseudo\sphinxhyphen{}norm = 1,
\begin{equation*}
\begin{split}\mathbf{U}(s) \cdot \mathbf{U}(s) = \gamma^2 \underbrace{\left( 1 - \frac{|\vec{v}|^2}{c^2} \right)}_{=\gamma^{-2}} = 1 \ .\end{split}
\end{equation*}
\sphinxAtStartPar
Using \(\tau\) as the parameter, \(\mathbf{U}\) has physical dimension of a velocity and pseudo\sphinxhyphen{}norm = \(c\).
\end{sphinxadmonition}

\sphinxAtStartPar
\sphinxstylestrong{4\sphinxhyphen{}acceleration} \(\mathbf{X}''(\tau)\) or \(\mathbf{X}''(s)\), \sphinxstylestrong{todo}


\section{Dynamics}
\label{\detokenize{ch/relativity-special/notes:dynamics}}
\sphinxAtStartPar
\sphinxstylestrong{4\sphinxhyphen{}momentum}
\begin{equation*}
\begin{split}\mathbf{P} = m \mathbf{U}\end{split}
\end{equation*}
\sphinxAtStartPar
Using Cartesian coordinates and \(\tau\) as independent variable,
\begin{equation*}
\begin{split}\mathbf{P} = m \mathbf{U} = m \frac{d \mathbf{X}}{d \tau} = m \gamma (c, \vec{v}) \ .\end{split}
\end{equation*}
\sphinxAtStartPar
The spatial component is \(\gamma\) times the 3\sphinxhyphen{}dimensional momentum \(\vec{p} = m \vec{v}\); the time component reads
\begin{equation*}
\begin{split}P^0 = m \gamma(w) c \ ,\end{split}
\end{equation*}
\sphinxAtStartPar
and for small ratio \(w:= \frac{v}{c}\) it can be expanded in Taylor series around \(w = 0\) as
\begin{equation*}
\begin{split}\gamma(w) \sim \gamma(0) + w \, \gamma'(0) + \frac{1}{2} \, w^2 \gamma''(0) + o(w^2) \ ,\end{split}
\end{equation*}
\sphinxAtStartPar
with
\begin{equation*}
\begin{split}\begin{aligned}
\left.\gamma(w)  \right|_{w=0} & = \left.\frac{1}{\sqrt{1 - w^2}}\right|_{w=0} = 1 \\
\left.\gamma'(w) \right|_{w=0} & = \left.-\frac{1}{2}(1 - w^2)^{-\frac{3}{2}} (- 2 w)\right|_{w=0} = w (1 - w^2)^{-\frac{3}{2}}= 0 \\
\left.\gamma''(w)\right|_{w=0} & = \left. \left( (1-w^2)^{-\frac{3}{2}} + w \left(-\frac{3}{2} \right)(1-w^2)^{-\frac{5}{2}} (- 2 w)  \right)\right|_{w=0} = \\ 
  & = \left. \left( (1-w^2)^{-\frac{3}{2}} + 3 w^2 (1-w^2)^{-\frac{5}{2}} \right)\right|_{w=0} = 1 \\
\end{aligned}\end{split}
\end{equation*}
\sphinxAtStartPar
and thus
\begin{equation*}
\begin{split}\gamma(w) = 1 + \frac{1}{2} w^2 + o(w^2)\end{split}
\end{equation*}
\sphinxAtStartPar
and
\begin{equation*}
\begin{split}\gamma(v/c) \, m \, c \sim m \, c \left( 1 + \frac{v^2}{c^2} \right) = \frac{1}{c} \left( mc^2 + \frac{1}{2} m |\vec{v}|^2 \right) \end{split}
\end{equation*}
\sphinxAtStartPar
Thus, recognizing energy (\(E = \gamma m c^2\)) and 3\sphinxhyphen{}momentum (\(\vec{p} = m_3 \vec{v}\), with \(m_3 := \gamma m\)), the 4\sphinxhyphen{}momentum can be written as
\begin{equation*}
\begin{split}\mathbf{P} = m \mathbf{U} = \gamma m \left( 1, \frac{\vec{v}}{c} \right) =: \frac{1}{c} \left( \frac{E}{c}, \vec{p} \right)\end{split}
\end{equation*}
\sphinxAtStartPar
Its pseudo\sphinxhyphen{}norm reads
\begin{equation*}
\begin{split}m^2 = \mathbf{P} \cdot \mathbf{P} = \frac{1}{c^4} \left( E^2 - c^2 |\vec{p}|^2 \right) \end{split}
\end{equation*}
\sphinxAtStartPar
and thus the relation between \(E\), \(\vec{p}\), \(m\) and \(c\),
\begin{equation*}
\begin{split}E^2 = m^2 c^4 + c^2 |\vec{p}|^2 \ ,\end{split}
\end{equation*}
\sphinxAtStartPar
from which, for \(\vec{v} = \vec{0} \rightarrow \vec{p} = \vec{0}\),
\begin{equation*}
\begin{split}E^2 = m^2 c^4 \ ,\end{split}
\end{equation*}
\sphinxAtStartPar
and keeping only the solution with positive energy (\sphinxstylestrong{todo} \sphinxstyleemphasis{reference to Dirac’s equation and anti\sphinxhyphen{}matter?})
\begin{equation*}
\begin{split}E = m c^2 \ .\end{split}
\end{equation*}

\subsection{Lagrangian approach}
\label{\detokenize{ch/relativity-special/notes:lagrangian-approach}}
\sphinxAtStartPar
\sphinxstylestrong{Free particle.}
\begin{equation*}
\begin{split}\mathbf{0} = \frac{d \mathbf{P}}{d s} = \frac{d }{d s} \left( m \mathbf{X}'(s) \right)\end{split}
\end{equation*}
\sphinxAtStartPar
Weak form
\begin{equation*}
\begin{split}\begin{aligned}
  0
  & = \mathbf{W}(s) \cdot \frac{d }{d s} \left( m \mathbf{X}'(s) \right) = \\
  & = \frac{d }{d s} \left[ m \mathbf{W}(s) \cdot \mathbf{X}'(s) \right] - m \mathbf{W}'(s) \cdot \mathbf{X}'(s)  = \\
\end{aligned}\end{split}
\end{equation*}
\sphinxAtStartPar
Using generalized coordinates \(q^k(s)\), the event can be written in parametric form as \(\mathbf{X}(q^k(s), s)\), while the velocity reads
\begin{equation*}
\begin{split}\mathbf{U}(s) = \mathbf{X}'(s) = \frac{d}{ds} \mathbf{X}(q^k(s), s) = {q^k}'(s) \underbrace{\frac{\partial \mathbf{X}}{\partial q^k}(q^k(s), s)}_{=\frac{\partial \mathbf{X}'}{\partial {q^k}'}} + \frac{\partial \mathbf{X}}{\partial s}(q^k(s), s) = \mathbf{U}({q^k}'(s), q^k(s), s)\end{split}
\end{equation*}
\sphinxAtStartPar
Choosing \(\mathbf{W} = \frac{\partial \mathbf{X}}{\partial q^k} = \frac{\partial \mathbf{X}'}{\partial {q^k}'}\) in the weak form,
\begin{equation*}
\begin{split}\begin{aligned}
0  
& = \frac{d }{d s} \left[ m \mathbf{W} \cdot \mathbf{X}' \right] - m \mathbf{W}' \cdot \mathbf{X}' = \\
& = \frac{d }{d s} \left[ m \frac{\partial \mathbf{X}'}{\partial {q^k}'} \cdot \mathbf{X}' \right] - m \dfrac{d}{ds} \frac{\partial \mathbf{X}}{\partial {q^k}} \cdot \mathbf{X}' = \\
& = \frac{1}{2} \left[ \frac{d}{d s} \left( \frac{\partial}{\partial {q^k}'}\left( m \mathbf{X}' \cdot \mathbf{X}' \right)  \right) - \frac{\partial}{\partial q^k}\left( m \mathbf{X}' \cdot \mathbf{X}' \right)  \right] =
\end{aligned}\end{split}
\end{equation*}
\sphinxAtStartPar
Defining
\begin{equation*}
\begin{split} f\left({q^k}'(s), q^k(s), s \right) = - m \mathbf{X}'\left({q^k}'(s), q^k(s), s\right) \cdot \mathbf{X}'\left({q^k}'(s), q^k(s), s\right) = - m \ ,\end{split}
\end{equation*}
\sphinxAtStartPar
multiplying by a “regular” generic function \(w(s)\), neglecting factor \(\frac{1}{2}\) and integrating by parts
\begin{equation*}
\begin{split}\begin{aligned}
 0
 & = - \int_{s=s_a}^{s_b} w(s) \left[ \frac{d}{ds} \frac{\partial f}{\partial {q^k}'} - \frac{\partial f}{\partial q^k} \right] \, ds = \\
 & = - \left.\left[ w(s) \frac{\partial f}{\partial {q^k}'} \right]\right|_{s=s_a}^{s_b} + \int_{s=s_a}^{s_b} \left[ w'(s) \frac{\partial f}{\partial {q^k}'} + \frac{\partial f}{\partial q^k} \right] \, ds = \\
 & = - \left.\left[ w(s) \frac{\partial f}{\partial {q^k}'} \right]\right|_{s=s_a}^{s_b} + \delta \int_{s=s_a}^{s_b} f\left( {q^k}'(s), q^k(s), s \right) \, ds \ .
\end{aligned}\end{split}
\end{equation*}
\sphinxAtStartPar
Thus, provided that \(w(s_1) = w(s_2) = 0\), equation of motion of free particle implies stationariety of functional
\begin{equation*}
\begin{split}\int_{s=s_a}^{s_b} f\left( {q^k}'(s), q^k(s), s \right) \, ds \ ,\end{split}
\end{equation*}
\sphinxAtStartPar
i.e.
\begin{equation*}
\begin{split}\delta \int_{s=s_a}^{s_b} f\left( {q^k}'(s), q^k(s), s \right) \, ds = 0\end{split}
\end{equation*}
\sphinxAtStartPar
Using \(t\) as independent parameter, \(ds = \gamma^{-1} \, c \, dt\), the functional can be recast as
\begin{equation*}
\begin{split}\int_{t=t_a}^{t_b}  -m \, c  \, \sqrt{1 - \frac{|\vec{v}|^2}{c^2}} \, dt \ ,\end{split}
\end{equation*}
\sphinxAtStartPar
to find the (3\sphinxhyphen{}dimensional) Lagrangian (multiply by \(c\) to get the right physical dimension; check if it’s required and wheter it’s possible to make \(c\) appear before),
\begin{equation*}
\begin{split}\mathscr{L} = - \sqrt{1 - \frac{|\vec{v}|^2}{c^2}} \, m \, c^2 \ ,\end{split}
\end{equation*}
\sphinxAtStartPar
and retrieve 3\sphinxhyphen{}momentum as (being \(\vec{v} = \dot{\vec{r}})\)
\begin{equation*}
\begin{split}\begin{aligned}
  \vec{p} 
  & := \frac{\partial \mathscr{L}}{\partial \dot{\vec{r}}} = \\
  &  = - m c^2 \frac{1}{2} \left(1-\frac{|\vec{v}|^2}{c^2} \right)^{-\frac{1}{2}} \left( - 2 \frac{\vec{v}}{c^2}\right) = \\
  &  = m \left(1-\frac{|\vec{v}|^2}{c^2} \right)^{-\frac{1}{2}} \vec{v} = \\
  &  = \gamma \, m \, \vec{v} \ ,
\end{aligned}\end{split}
\end{equation*}
\sphinxAtStartPar
and energy as
\begin{equation*}
\begin{split}\begin{aligned}
  E
  & := \vec{p} \cdot \vec{v} - \mathscr{L} = \\
  & = \gamma \, m \, |\vec{v}|^2 + \gamma^{-1} \, m \, c^2 = \\
  & = \gamma \, m \, c^2 \left( \frac{|\vec{v}|^2}{c^2} + \gamma^{-2} \right) = \\
  & = \gamma \, m \, c^2 \left( \frac{|\vec{v}|^2}{c^2} + 1 - \frac{|\vec{v}|^2}{c^2} \right) = \\
  & = \gamma \, m \, c^2 \ .
\end{aligned}\end{split}
\end{equation*}

\section{Electromagnetism}
\label{\detokenize{ch/relativity-special/notes:electromagnetism}}

\subsection{Classical electromagnetic theory}
\label{\detokenize{ch/relativity-special/notes:classical-electromagnetic-theory}}

\subsubsection{Maxwell equations}
\label{\detokenize{ch/relativity-special/notes:maxwell-equations}}
\sphinxAtStartPar
Maxwell equations read
\begin{equation*}
\begin{split}\begin{cases}
\nabla \cdot \vec{d} = \rho_f \\
\nabla \times \vec{e} + \partial_t \vec{b} = \vec{0} \\
\nabla \cdot \vec{b} = 0 \\
\nabla \times \vec{h} - \partial_t \vec{d} = \vec{j}_f
\end{cases}\end{split}
\end{equation*}
\sphinxAtStartPar
or in vacuum, with \(\rho_f = \rho\), \(\vec{j} = \vec{j}_f\), \(\vec{d} = \varepsilon_0 \vec{e}\), \(\vec{b} = \mu_0 \vec{h}\)
\begin{equation*}
\begin{split}\begin{cases}
\nabla \cdot \vec{e} = \frac{\rho}{\varepsilon_0} \\
\nabla \times \vec{e} + \partial_t \vec{b} = \vec{0} \\
\nabla \cdot \vec{b} = 0 \\
\nabla \times \vec{b} - \mu_0 \varepsilon_0 \partial_t \vec{e} = \mu_0 \vec{j}
\end{cases}\end{split}
\end{equation*}

\subsubsection{Electromagnetic potentials}
\label{\detokenize{ch/relativity-special/notes:electromagnetic-potentials}}
\sphinxAtStartPar
The electromagnetic field can be written in terms of the electromagnetic potentials
\begin{equation*}
\begin{split}\begin{cases}
 \vec{b} = \nabla \times \vec{a} \\
 \vec{e} = -\partial_t \vec{a} - \nabla \varphi \\
\end{cases}\end{split}
\end{equation*}

\subsubsection{Lorentz force}
\label{\detokenize{ch/relativity-special/notes:lorentz-force}}
\sphinxAtStartPar
A particle in motion in a electromagnetic field is subject to Lorentz force. In classical electromagnetism, the expression of Lorentz force reads
\begin{equation*}
\begin{split}\vec{F} = q \left( \vec{e} - \vec{b} \times \vec{v} \right) \ ,\end{split}
\end{equation*}
\sphinxAtStartPar
whose power is
\begin{equation*}
\begin{split}\vec{v} \cdot \vec{F} = \vec{v} \cdot q \left( \vec{e} - \vec{b} \times \vec{v} \right) = q \vec{v} \cdot \vec{e} \ . \end{split}
\end{equation*}

\subsection{Electromagnetic potential}
\label{\detokenize{ch/relativity-special/notes:electromagnetic-potential}}\begin{equation*}
\begin{split}\begin{cases}
 \vec{b} = \nabla \times \vec{a} \\
 \vec{e} = -\partial_t \vec{a} - \nabla \varphi \\
\end{cases}\end{split}
\end{equation*}\begin{equation*}
\begin{split}\mathbf{A} = \mathbf{E}_{\alpha} A^{\alpha} = \frac{\varphi}{c} \mathbf{E}_0 + \vec{a}\end{split}
\end{equation*}\begin{equation*}
\begin{split}\symbf{\nabla} \mathbf{A} =
\left( \mathbf{E}^{\alpha} \frac{\partial}{\partial X^{\alpha}} \right) \left( A^{\beta} \mathbf{E}_{\beta} \right)
= \frac{\partial A^{\beta}}{\partial X^{\alpha}} \mathbf{E}^{\alpha} \otimes \mathbf{E}_{\beta} 
= g_{\alpha \gamma} \frac{\partial A^{\beta}}{\partial X^{\alpha}} \mathbf{E}_{\gamma} \otimes \mathbf{E}_{\beta}  \ .\end{split}
\end{equation*}
\sphinxAtStartPar
whose components may be collected in a 2\sphinxhyphen{}dimensional array (first index for rows, second index for columns),
\begin{equation*}
\begin{split}(\nabla \mathbf{A})_{\alpha}^{\beta} = \frac{\partial A^{\beta}}{\partial X^{\alpha}} =
\begin{bmatrix}
 c^{-2} \partial_t \varphi & c^{-1}\partial_x \varphi & c^{-1}\partial_y \varphi & c^{-1}\partial_z \varphi \\
 c^{-1} \partial_t a_x     &       \partial_x a_x     &       \partial_y a_x     &       \partial_z a_x     \\
 c^{-1} \partial_t a_y     &       \partial_x a_y     &       \partial_y a_y     &       \partial_z a_y     \\
 c^{-1} \partial_t a_z     &       \partial_x a_z     &       \partial_y a_z     &       \partial_z a_z     \\
\end{bmatrix}\end{split}
\end{equation*}
\sphinxAtStartPar
or covariant\sphinxhyphen{}covariant coomponents,
\begin{equation*}
\begin{split}(\nabla \mathbf{A})_{\alpha \beta} = \frac{\partial A_{\beta}}{\partial X^{\alpha}} = g_{\beta \gamma} \frac{\partial A^{\gamma}}{\partial X^{\alpha}} = 
\begin{bmatrix}
  c^{-2}\partial_t \varphi & c^{-1}\partial_x \varphi & c^{-1}\partial_y \varphi & c^{-1}\partial_z \varphi \\
 -c^{-1}\partial_t a_x     &      -\partial_x a_x     &      -\partial_y a_x     &      -\partial_z a_x     \\
 -c^{-1}\partial_t a_y     &      -\partial_x a_y     &      -\partial_y a_y     &      -\partial_z a_y     \\
 -c^{-1}\partial_t a_z     &      -\partial_x a_z     &      -\partial_y a_z     &      -\partial_z a_z     \\
\end{bmatrix}\end{split}
\end{equation*}
\sphinxAtStartPar
or contravariant\sphinxhyphen{}contravariant coomponents,
\begin{equation*}
\begin{split}(\nabla \mathbf{A})^{\alpha \beta} = \frac{\partial A^{\beta}}{\partial X_{\alpha}} = g_{\beta \gamma} \frac{\partial A^{\alpha}}{\partial X^{\gamma}} = 
\begin{bmatrix}
  c^{-2}\partial_t \varphi &-c^{-1}\partial_x \varphi &-c^{-1}\partial_y \varphi &-c^{-1}\partial_z \varphi \\
  c^{-1}\partial_t a_x     &      -\partial_x a_x     &      -\partial_y a_x     &      -\partial_z a_x     \\
  c^{-1}\partial_t a_y     &      -\partial_x a_y     &      -\partial_y a_y     &      -\partial_z a_y     \\
  c^{-1}\partial_t a_z     &      -\partial_x a_z     &      -\partial_y a_z     &      -\partial_z a_z     \\
\end{bmatrix}\end{split}
\end{equation*}
\sphinxAtStartPar
The electromagnetic field tensor is defined as the anti\sphinxhyphen{}symmetric part of the gradient of the 4\sphinxhyphen{}electromagnetic potential,
\begin{equation*}
\begin{split}\mathbf{F} = \left[ \symbf{\nabla} \mathbf{A} - \left( \symbf{\nabla} \mathbf{A} \right)^T \right]\end{split}
\end{equation*}
\sphinxAtStartPar
whose components may be collected in a 2\sphinxhyphen{}dimensional array (first index for rows, second index for columns),
\begin{equation*}
\begin{split}F^{\alpha \beta} = 
\begin{bmatrix}
                        0 & -\frac{\underline{e}^T}{c} \\
  \frac{\underline{e}}{c} & \underline{b}_{\times}
\end{bmatrix}
\qquad , \qquad 
F_{\alpha \beta} = 
\begin{bmatrix}
                        0 & \frac{\underline{e}^T}{c} \\
 -\frac{\underline{e}}{c} & \underline{b}_{\times}
\end{bmatrix}
\end{split}
\end{equation*}

\subsection{Electromagnetic field and electromagnetic field equations}
\label{\detokenize{ch/relativity-special/notes:electromagnetic-field-and-electromagnetic-field-equations}}
\sphinxAtStartPar
The pair of Maxwell equations
\begin{equation*}
\begin{split}\begin{cases}
  \rho_f    = \nabla \cdot \vec{d}  \\
  \vec{j}_f = - \partial_t \vec{d} + \nabla \times \vec{h}  \\
\end{cases}\end{split}
\end{equation*}
\sphinxAtStartPar
can be re\sphinxhyphen{}written in \(4\)\sphinxhyphen{}formalism, using \(4\)\sphinxhyphen{}gradient in Cartesian coordinates
\begin{equation*}
\begin{split}\symbf{\nabla} = \mathbf{E}^{\alpha} \frac{\partial}{\partial X^{\alpha}}
 = \mathbf{E}_0 \, \frac{\partial }{c \partial t} + \mathbf{E}_i \frac{\partial}{\partial x^i}
 = \mathbf{E}_0 \, \frac{\partial }{c \partial t} + \nabla \ ,\end{split}
\end{equation*}
\sphinxAtStartPar
and the definition of the 4\sphinxhyphen{}current density vector
\begin{equation*}
\begin{split}\mathbf{J} = J^{\alpha} \mathbf{E}_{\alpha} = c \rho \, \mathbf{E}_0 + \vec{j}\end{split}
\end{equation*}
\sphinxAtStartPar
so that
\begin{equation*}
\begin{split}\begin{aligned}
  c \rho \mathbf{E}_0 + \vec{j} = \symbf{\nabla} \cdot \mathbf{F} 
  = \symbf{\nabla} \cdot
   [ & ( 0\, \mathbf{E}_0 + c \vec{d} ) \otimes \mathbf{E}_0 + ( - \mathbf{E}_0 c \vec{d} + \vec{h}_{\times} ) ] \ ,
\end{aligned}\end{split}
\end{equation*}
\sphinxAtStartPar
with the displacement field tensor,
\begin{equation*}
\begin{split}\mathbf{D} = D^{\alpha \beta} \, \mathbf{E}_{\alpha} \, \mathbf{E}_{\beta} \ , \end{split}
\end{equation*}
\sphinxAtStartPar
with components (rows for the first index, columns for the second index)
\begin{equation*}
\begin{split}D^{\alpha \beta} = 
  \begin{bmatrix}     0 & -c d_x & - c d_y & - c d_z \\
                  c d_x &      0 & -   h_z &     h_y \\
                  c d_y &    h_z &       0 & -   h_x \\
                  c d_z & -  h_y &     h_x &       0 \\
  \end{bmatrix} =
  \begin{bmatrix} 0 & - c \underline{d}^T \\ c \underline{d} & \underline{h}_{\times} \ .
  \end{bmatrix}
\end{split}
\end{equation*}


\sphinxAtStartPar
The pair of Maxwell equations
\begin{equation*}
\begin{split}\begin{cases}
  \nabla \cdot \vec{b} = 0 \\
  \partial_t \vec{b} + \nabla \times \vec{e} = \vec{0} \\
\end{cases}\end{split}
\end{equation*}
\sphinxAtStartPar
can be re\sphinxhyphen{}written in \(4\)\sphinxhyphen{}formalism as
\begin{equation*}
\begin{split}0 = \partial_{\mu} F_{\eta \xi} + \partial_{\eta} F_{\xi \mu} + \partial_{\xi} F_{\mu \eta} \end{split}
\end{equation*}
\sphinxAtStartPar
Among these \(64 = 4^3\) equations, there are only 4 independent equations.
\begin{itemize}
\item {} 
\sphinxAtStartPar
If 2 indices are the same, the corresponding equation is the identity \(0 = 0\). As an example, if \(\mu = \eta\)
\begin{equation*}
\begin{split}0 =  \partial_{\mu} F_{\mu \xi} + \partial_{\mu} \underbrace{F_{\xi \mu}}_{-F_{\mu xi}} + \partial_{\xi} \underbrace{F_{\mu \mu}}_{=0} = 0 \ , \end{split}
\end{equation*}
\sphinxAtStartPar
thus only combinations with different indices may provide some information.

\item {} 
\sphinxAtStartPar
Given the ordered set of indices \((\mu, \eta, \xi)\), switching a pair of indices provides the same equation. As an example, switching \(\mu\) and \(\eta\)
\begin{equation*}
\begin{split}\begin{aligned}
    0 & = \partial_{\eta} F_{\mu  \xi} + \partial_{\mu} F_{\xi \eta} + \partial_{\xi} F_{\eta \mu} = \\
      & = \partial_{\eta} ( - F_{\xi \mu} ) + \partial_{\mu} ( - F_{\eta \xi} ) + \partial_{\xi} ( -F_{\mu \eta} ) \ .
  \end{aligned}\end{split}
\end{equation*}
\item {} 
\sphinxAtStartPar
Thus, only 4 combination of different indices, without taking order into account, provide independent information
\begin{equation*}
\begin{split}\begin{aligned}
    (1,2,3): & \quad 0 = \partial_{1} F_{23} + \partial_{2} F_{31} + \partial_{3} F_{12} = \partial_x (-b_x) + \partial_y (-b_y) + \partial_z (-b_z)  \\
    (2,3,0): & \quad 0 = \partial_{2} F_{30} + \partial_{3} F_{02} + \partial_{0} F_{23} = \partial_y \left(-\frac{e_z}{c} \right) + \partial_z \left( \frac{e_y}{c} \right) + \partial_{ct} (-b_x) \\
    (3,0,1): & \quad 0 = \partial_{3} F_{01} + \partial_{0} F_{13} + \partial_{1} F_{30} = \partial_z \left(\frac{e_x}{c}\right) + \partial_{ct} (-b_y) + \partial_x  \left(-\frac{e_z}{c}\right) \\
    (0,1,2): & \quad 0 = \partial_{0} F_{12} + \partial_{1} F_{20} + \partial_{2} F_{01} = \partial_{ct} (-b_z) + \partial_x \left(-\frac{e_y}{c}\right) + \partial_y  \left(\frac{e_x}{c}\right)     \\
  \end{aligned}\end{split}
\end{equation*}
\sphinxAtStartPar
i.e.
\begin{equation*}
\begin{split}\begin{aligned}
    (1,2,3): & \quad 0 = - \nabla \cdot \vec{b} \\
    (2,3,0): & \quad 0 = -\frac{1}{c} \left[ \partial_t b_x + (\partial_y e_z - \partial_z e_y ) \right] \\ 
    (3,0,1): & \quad 0 = -\frac{1}{c} \left[ \partial_t b_y + (\partial_z e_x - \partial_x e_z ) \right] \\
    (0,1,2): & \quad 0 = -\frac{1}{c} \left[ \partial_t b_z + (\partial_x e_y - \partial_y e_x ) \right] \\
  \end{aligned}\end{split}
\end{equation*}
\sphinxAtStartPar
i.e.
\begin{equation*}
\begin{split}\begin{cases}
        0  = \nabla \cdot \vec{b} \\
   \vec{0} = \partial_t \vec{b} + \nabla \times \vec{e} \\
  \end{cases}\end{split}
\end{equation*}
\end{itemize}


\subsection{Point particle in electromagnetic field}
\label{\detokenize{ch/relativity-special/notes:point-particle-in-electromagnetic-field}}
\sphinxAtStartPar
Lorentz 4\sphinxhyphen{}force acting on a point charge of electric charge charge \(q\) reads
\begin{equation*}
\begin{split}\mathbf{f} = \mathbf{F} \cdot \mathbf{J} = q \, \mathbf{F} \cdot \mathbf{U} \ .\end{split}
\end{equation*}
\sphinxAtStartPar
so that the dynamical equation reads
\begin{equation*}
\begin{split}m \, \mathbf{X}'' = q \, \mathbf{F} \cdot \mathbf{X}'\end{split}
\end{equation*}

\subsection{Energy balance}
\label{\detokenize{ch/relativity-special/notes:energy-balance}}\begin{equation*}
\begin{split}\begin{aligned}
  \frac{\partial u      }{\partial t} & = \\
  \frac{\partial \vec{s}}{\partial t} & = \\
\end{aligned}\end{split}
\end{equation*}
\sphinxAtStartPar
…
\begin{equation*}
\begin{split}\symbf{\nabla} \cdot \mathbf{T} = - \mathbf{F} \cdot \mathbf{J}\end{split}
\end{equation*}
\sphinxstepscope


\part{General Relativity}

\sphinxstepscope


\chapter{General Relativity}
\label{\detokenize{ch/relativity-general/intro:general-relativity}}\label{\detokenize{ch/relativity-general/intro:relativity-general-intro}}\label{\detokenize{ch/relativity-general/intro::doc}}
\sphinxstepscope


\chapter{General Relativity \sphinxhyphen{} Notes}
\label{\detokenize{ch/relativity-general/notes:general-relativity-notes}}\label{\detokenize{ch/relativity-general/notes:relativity-general-notes}}\label{\detokenize{ch/relativity-general/notes::doc}}
\sphinxstepscope


\part{Statistical Mechanics}

\sphinxstepscope


\chapter{Statistical Physics}
\label{\detokenize{ch/statistical-mechanics/intro:statistical-physics}}\label{\detokenize{ch/statistical-mechanics/intro:statistical-mechanics-intro}}\label{\detokenize{ch/statistical-mechanics/intro::doc}}
\sphinxstepscope


\chapter{Statistical Physics \sphinxhyphen{} Notes}
\label{\detokenize{ch/statistical-mechanics/notes:statistical-physics-notes}}\label{\detokenize{ch/statistical-mechanics/notes:statistical-mechanics-notes}}\label{\detokenize{ch/statistical-mechanics/notes::doc}}

\section{Ensembles}
\label{\detokenize{ch/statistical-mechanics/notes:ensembles}}\label{\detokenize{ch/statistical-mechanics/notes:statistical-mechanics-notes-ensembles}}

\section{Microcanonical ensemble}
\label{\detokenize{ch/statistical-mechanics/notes:microcanonical-ensemble}}\label{\detokenize{ch/statistical-mechanics/notes:id1}}

\section{Canonical ensemble}
\label{\detokenize{ch/statistical-mechanics/notes:canonical-ensemble}}\label{\detokenize{ch/statistical-mechanics/notes:id2}}

\section{Macrocanonical esemble}
\label{\detokenize{ch/statistical-mechanics/notes:macrocanonical-esemble}}\label{\detokenize{ch/statistical-mechanics/notes:id3}}

\section{Statistics}
\label{\detokenize{ch/statistical-mechanics/notes:statistics}}\label{\detokenize{ch/statistical-mechanics/notes:statistical-mechanics-notes-distributions}}
\sphinxAtStartPar
Each of the \(N\) components of the system is in an \sphinxstylestrong{energy level} \(i\). Energy level \(i\) has \(g_i\) sublevels with the same energy level.
\begin{itemize}
\item {} 
\sphinxAtStartPar
energy levels, \(E_i\) of each component

\item {} 
\sphinxAtStartPar
occupation number \(N_i\) of level \(i\)

\item {} 
\sphinxAtStartPar
\sphinxstylestrong{Central role of energy.} In a system macroscopically at rest, the energy of a system is the only macroscopic meaningful non\sphinxhyphen{}zero mechanical quantity, constant for closed and isolated systems

\item {} 
\sphinxAtStartPar
\sphinxstylestrong{Principle of maximum uncertainty}, \sphinxstylestrong{maximum entropy}, \sphinxstylestrong{minimum information}: given a measurement of a macroscopic variable \(V\), describing the macrostate of the system, the feasible un\sphinxhyphen{}observed/able microstates of the system are the microstates consistent with it: there’s usually a sharp maximum of in the probability density of the micrsotates.

\end{itemize}

\sphinxAtStartPar
Given a macrostate, what’s the number of ways \(W(N_i; g_i)\) to get a consistent microstate? Once the expression is found, constrained optimization follows: optimization w.r.t. \(N_i\) is usually performed in the limit of \(N_i \rightarrow +\infty\) (why in Fermi\sphinxhyphen{}Dirac distribution, obeying Pauli exclusion principle?), with the values of the macroscopic variables as constraints usually treated with Lagrange multiplier.


\subsection{Maxwell\sphinxhyphen{}Boltzmann}
\label{\detokenize{ch/statistical-mechanics/notes:maxwell-boltzmann}}\label{\detokenize{ch/statistical-mechanics/notes:statistical-mechanics-notes-distributions-mb}}
\sphinxAtStartPar
Statistics of distinguishible components.


\subsection{Bose\sphinxhyphen{}Einstein}
\label{\detokenize{ch/statistical-mechanics/notes:bose-einstein}}\label{\detokenize{ch/statistical-mechanics/notes:statistical-mechanics-notes-distributions-be}}
\sphinxAtStartPar
Statistics of undistinguishable components that can be in the same (sub)level.
Given the number of elementary components \(\sum_{i} N_i = N\) and the energy \(\sum_{i} N_i E_i = E\),
\begin{equation}\label{equation:ch/statistical-mechanics/notes:eq:be}
\begin{split}W_{BE,i} = \frac{(N_i + g_i - 1)!}{N_i! (g_i-1)!} \qquad , \qquad W_{BE} = \prod_i W_{BE,i} \ .\end{split}
\end{equation}\subsubsection*{Counting microstates}

\sphinxAtStartPar
\sphinxstylestrong{todo} write page \sphinxstyleemphasis{Combinatorics} and add link

\sphinxAtStartPar
\sphinxstylestrong{Most likely microstate.} Instead of maximizing \eqref{equation:ch/statistical-mechanics/notes:eq:be}, the objective function is \(\ln W_{BE}\), after using Stirling approximation in the limit of large \(N_i\) and \(g_i\), \(N_i! \sim \left(\frac{N_i}{e} \right)^{N_i}\). The approximate occupation number of one of the \(G_i\) sublevels of the \(i^{th}\) level of the most likely microstate is
\begin{equation*}
\begin{split}n_i := \frac{N_i}{G_i} = \frac{1}{e^{\alpha + \beta E_i} - 1} \ .\end{split}
\end{equation*}\subsubsection*{Optimization}
\begin{equation*}
\begin{split}\begin{aligned}
  J(N_i, \alpha, \beta)
  & = \ln W_{BE} + \alpha \left( N - \sum_i N_i \right) + \beta \left(E - \sum_i N_i E_i \right) = \\
  & = \sum_i \left\{ \ln(N_i + g_i -1)! - \ln N_i! - \ln (g_i-1)! \right\} + \alpha \left( N - \sum_i N_i \right) + \beta \left(E - \sum_i N_i E_i \right) \simeq \\
  & \simeq \sum_i \left\{ (N_i+g_i-1)\ln(N_i+g_i-1) - N_i\ln N_i - (g_i-1) \ln (g_i-1) + N_i + g_i -1 - N_i - (g_i-1) \right\} + \alpha \left( N - \sum_i N_i \right) + \beta \left(E - \sum_i N_i E_i \right) = \\
  & = \sum_i \left\{ (N_i+g_i-1)\ln(N_i+g_i-1) - N_i\ln N_i - (g_i-1) \ln (g_i-1) \right\} + \alpha \left( N - \sum_i N_i \right) + \beta \left(E - \sum_i N_i E_i \right) \\
\end{aligned}\end{split}
\end{equation*}
\sphinxAtStartPar
Using \(\partial_{n} (n+a) \ln (n+a) = \ln (n+a) + 1\),
\begin{equation*}
\begin{split}\begin{aligned}
  0 = \partial_{N_k} J 
  & \simeq \left\{ \ln (N_k+g_k-1) - \ln N_k \right\} - \alpha - \beta E_k \ ,
\end{aligned}\end{split}
\end{equation*}
\sphinxAtStartPar
and thus
\begin{equation*}
\begin{split}\ln \frac{N_k + g_k - 1}{N_k} = \alpha + \beta E_k \ ,\end{split}
\end{equation*}\begin{equation*}
\begin{split}\frac{N_k + g_k - 1}{N_i} = e^{\alpha + \beta E_k}\end{split}
\end{equation*}\begin{equation*}
\begin{split}N_k = \frac{g_k - 1}{e^{\alpha + \beta E_k}-1} \simeq \frac{g_k}{e^{\alpha + \beta E_k}-1} \ , \end{split}
\end{equation*}
\sphinxAtStartPar
Thus, in the limit of \(g_k \gg 1\), the  occupation number of the \(k\) level is
\begin{equation*}
\begin{split}N_k = \frac{G_k}{e^{\alpha + \beta E_k} - 1} \ ,\end{split}
\end{equation*}
\sphinxAtStartPar
and the average occupation number of one of the \(g_k\) sublevels in the \(k\) level is
\begin{equation*}
\begin{split}n_k := \frac{N_k}{G_k} = \frac{1}{e^{\alpha + \beta E_k} - 1}\end{split}
\end{equation*}\subsubsection*{\sphinxstylestrong{Meaning of \protect\(\alpha\protect\), \protect\(\beta\protect\)}}
\label{ch/statistical-mechanics/notes:example-0}
\begin{sphinxadmonition}{note}{Example 1 (Black\sphinxhyphen{}body radiation: Planck, Wien, and Stefan\sphinxhyphen{}Boltzmann laws)}



\sphinxAtStartPar
\sphinxstylestrong{Planck’s law.} Energy density w.r.t. frequency
\begin{equation*}
\begin{split}u_{f}(f, T) = \frac{8 \pi h f^3}{c^3} \frac{1}{e^{\frac{hf}{k_B T}} - 1}\end{split}
\end{equation*}\subsubsection*{Planck’s law in a cubic box}

\sphinxAtStartPar
Planck’s law uses:
\begin{itemize}
\item {} 
\sphinxAtStartPar
relation between pulsation and wave vector, or frequency and wave number and the speed of light \(c\) for light waves
\begin{equation*}
\begin{split}c = \frac{\omega}{|\vec{k}|} = \lambda f\end{split}
\end{equation*}\begin{equation*}
\begin{split}f = \frac{\omega}{2\pi} = \frac{c |\vec{k}|}{2 \pi}\end{split}
\end{equation*}
\item {} 
\sphinxAtStartPar
Planck assumption that the minimum non\sphinxhyphen{}zero energy of a mode with frequency \(f\) is \(E = h f\), and all the possible values of the energy of the mode is
\begin{equation*}
\begin{split}E_m = m h f \quad , \quad m \in \mathbb{N} \ .\end{split}
\end{equation*}
\end{itemize}

\sphinxAtStartPar
Taking a cubic box with sides \(L_x = L_y = L_z = L\), the possibile modes have (\sphinxstylestrong{todo} \sphinxstyleemphasis{why? Which boundary condition? Periodic? Some physical? Just fictitious discretization?}) in each direction wave\sphinxhyphen{}lengths \(\lambda_n = \frac{L}{|\vec{n}|} = \frac{2 \pi}{|\vec{k}|}\),
\begin{equation*}
\begin{split}\vec{k} = \frac{2 \pi}{L} \vec{n} \ .\end{split}
\end{equation*}
\sphinxAtStartPar
Mode density in \(\vec{n}\)\sphinxhyphen{}domain is 2 mode per each volume of unit length (2 polarization), and thus the number of modes \(d N\) in an elementary volume is
\begin{equation*}
\begin{split}d N = 2 \, d^3 \vec{n} \ ,\end{split}
\end{equation*}
\sphinxAtStartPar
Changing variables, it’s possible to find the mode density w.r.t. wave vector \(\vec{k}\),
\begin{equation*}
\begin{split}d N = 2 \, d^3 \vec{n} = 2 \, \frac{L^3}{(2 \pi)^3} \, d^3 \vec{k} \ ,\end{split}
\end{equation*}
\sphinxAtStartPar
or with its absolute value, exploiting the isotropy of the density function \sphinxhyphen{} and writing the elementary volume using “spherical coordinates” \(d^3 \vec{k} = 4 \pi \left| \vec{k} \right|^2 d \, \left| \vec{k} \right|\),
\begin{equation*}
\begin{split}\begin{aligned}
  d N
  & = \frac{V}{(2 \pi)^3} 8 \pi \left| \vec{k} \right|^2 d \left| \vec{k} \right| = \\
  & = \frac{V}{(2 \pi)^3} 8 \pi \frac{8 \pi^3}{c^3} f^2 d f = \\
  & = V \frac{8 \pi}{c^3} f^2 df =: V g(f) df \ .
\end{aligned}\end{split}
\end{equation*}\subsubsection*{Average energy of a mode}

\sphinxAtStartPar
Using Boltzmann distribution (\sphinxstylestrong{why?}) for the energy distribution in a single mode,
\begin{equation*}
\begin{split}P(E_r) = \frac{e^{-\beta E_r}}{Z} \ ,\end{split}
\end{equation*}
\sphinxAtStartPar
with \(E_r = r h f\), and the partition function
\begin{equation*}
\begin{split}Z = \sum_{s} e^{- \beta E_s} = \sum_s e^{-\beta h f s} = \frac{1}{1 - e^{-\beta h f}} \ .\end{split}
\end{equation*}
\sphinxAtStartPar
The average energy of the mode reads
\begin{equation*}
\begin{split}\begin{aligned}
  \langle E \rangle
  & = \sum_r E_r P(E_r) = \\
  & = \sum_r r h f \frac{e^{- \beta h f r}}{Z} = \\
  & = h f (1-e^{-\beta h f}) \sum_r r e^{- \beta h f r} = \\
  & = h f (1-e^{-\beta h f}) \frac{e^{- \beta h f}}{(1-e^{-\beta h f})^2} = \\
  & = \frac{h f}{e^{\beta h f} - 1}  \ .
\end{aligned}\end{split}
\end{equation*}
\sphinxAtStartPar
Putting together the mode number density and the average energy of a mode, the energy density per unit volume, per frequency reads
\begin{equation*}
\begin{split}\begin{aligned}
  u(f, T)
  & = \langle E \rangle(f) \, g(f) = \\
  & = \frac{hf}{e^{\beta h f} - 1} \frac{8 \pi}{c^3} f^2 = \\
  & = \frac{8 \pi h f^3}{c^3} \frac{1}{e^{\beta h f} - 1} \ .
\end{aligned}\end{split}
\end{equation*}\subsubsection*{Property of the series}
\begin{equation*}
\begin{split}\sum_{n=0}^{+\infty} n x^n = \frac{x}{(1-x)^2}\end{split}
\end{equation*}
\sphinxAtStartPar
\sphinxstylestrong{Proof.} If the series is convergent (is this the required condition?)
\begin{equation*}
\begin{split}\frac{d}{d x} \sum_{n=0}^{+\infty} x^n = \frac{d}{dx} \frac{1}{1 - x} = \frac{1}{(1-x)^2}\end{split}
\end{equation*}\begin{equation*}
\begin{split}\frac{d}{d x} \sum_{n=0}^{+\infty} x^n = \sum_{n=0}^{+\infty} n x^{n-1}\end{split}
\end{equation*}\begin{equation*}
\begin{split}x \frac{d}{d x} \sum_{n=0}^{+\infty} x^n = \sum_{n=0}^{+\infty} n x^n = \frac{x}{(1-x)^2}\end{split}
\end{equation*}
\sphinxAtStartPar
\sphinxstylestrong{Sperctral radiance}, \(B_{f}\), so that an infinitesimal amount of power radiated by a surface … is \(d P = B_f(f,T) \cos \theta \, dA \, d\Omega \, d f\)
\begin{equation*}
\begin{split}B_{f}(f, T) = \frac{2 h f^3}{c^2}\frac{1}{e^{\frac{hf}{k_B T}} - 1} \ .\end{split}
\end{equation*}
\sphinxAtStartPar
This expression is obtained%
\begin{footnote}[1]\sphinxAtStartFootnote
\sphinxhref{https://nanohub.org/wiki/DerivationofPlancksLaw}{Derivation of Planck’s Law}.
%
\end{footnote} assuming homogeneous radiation from a small hole cut into a wall of the box. Only half of the energy radiates through the hole \sphinxhyphen{} so factor \(\frac{1}{2}\) in front of the energy density \sphinxhyphen{} through a solid angle \(2 \pi\) \sphinxhyphen{} and thus this process give the same result as a radiation of all the energy density in all the space directions, just providing the same factor \(\frac{1}{4 \pi}\). The flux of energy “has velocity” \(c\) and thus
\begin{equation*}
\begin{split}B_{f}(f, T) = \frac{1}{4 \pi} u_{f}(f,T) c \ .\end{split}
\end{equation*}
\sphinxAtStartPar
\sphinxstylestrong{Wien’s law.} Wien’s law tells that the frequency \(f^*\) corresponding to the maximum of the spectral radiance of a black\sphinxhyphen{}body radiation described by Planck’s law is proportional to its temperature.

\sphinxAtStartPar
From direct evaluation of the derivative of the spectral radiance as a function of \(f\),
\begin{equation*}
\begin{split}\begin{aligned}
  \partial_f B_{f}(f,T)
  & = \frac{2 h}{c^2} \left[ 3 f^2 \frac{1}{e^{\frac{hf}{k_B T}}-1} + f^3 \left(-\frac{\frac{h}{k_B T} e^{\frac{hf}{k_B T}}}{\left( e^{\frac{hf}{k_B T}} - 1 \right)^2}  \right) \right] = \\
  & = \frac{2 h f^2 e^{\frac{hf}{k_B T}}}{c^2 \left( e^{\frac{hf}{k_B T}} - 1 \right)^2} \left[ 3 \left( 1 - e^{-\frac{hf}{k_B T}} \right) - \frac{h f}{k_B T}  \right] \ .
\end{aligned}\end{split}
\end{equation*}
\sphinxAtStartPar
Now, if \(\partial_f B_{f}(f,T) = 0\) the frequency is either \(f = 0\), or the solution of the nonlinear algebraic equation
\begin{equation*}
\begin{split}0 = 3 \left(1 - e^{-\frac{h f}{k_B T}} \right) - \frac{hf}{k_B T} \ .\end{split}
\end{equation*}
\sphinxAtStartPar
Defining \(x := \frac{h f}{k_B T}\), this equation becomes
\begin{equation*}
\begin{split}0 = 3 (1 - e^x) - x \ ,\end{split}
\end{equation*}
\sphinxAtStartPar
whose solution \(x^* \approx 2.82\) can be easily evaluated with an iterative method (or expressed in term of the Lambert’s function \(W\), so loved at Stanford and on Youtube: they’d probaly like to look at tabulated values, or pose). Once the solution \(x^*\) of this non\sphinxhyphen{}dimensional equation is found, the frequency where maximum energy density occurs reads
\begin{equation*}
\begin{split}f^* = \frac{k_B T}{h} x^* \simeq 2.82 \frac{k_B}{h} T \ .\end{split}
\end{equation*}
\sphinxAtStartPar
\sphinxstylestrong{Stefan\sphinxhyphen{}Boltzmann law.}
\begin{equation*}
\begin{split}\begin{aligned}
  \frac{P}{A} 
  & = \int B_{f}(f,T) \cos \phi \, df \, d\Omega = \\
  & = \int_{f=0}^{+\infty} \int_{\phi = 0}^{\frac{\pi}{2}} \int_{\theta=0}^{2\pi} B_{f}(f,T) \cos \phi \sin \phi \, df \, d\phi \, d \theta = \\
  & = \pi \int_{f=0}^{+\infty} B_{f}(f,T) \, d f = \\
  & = \frac{2 \pi h}{c^2} \int_{f=0}^{+\infty} \frac{f^3}{e^{\frac{hf}{k_B T}} - 1} \, d f = \\
  & = \frac{2 \pi h}{c^2} \left( \frac{k_B T}{h} \right)^4 \int_{u=0}^{+\infty} \frac{u^3}{e^u - 1} \, d u \ .
\end{aligned}\end{split}
\end{equation*}
\sphinxAtStartPar
The value of the integral is \(\frac{\pi^4}{15}\) and thus
\begin{equation*}
\begin{split}\frac{P}{A} = \sigma T^4 \qquad , \qquad \sigma = \frac{2 \pi^5 k_B^4}{15 c^2 h^3} \ .\end{split}
\end{equation*}\end{sphinxadmonition}
\label{ch/statistical-mechanics/notes:example-1}
\begin{sphinxadmonition}{note}{Example 2 (Energy density and radiance)}



\sphinxAtStartPar
\sphinxstylestrong{Radiance.} The radiance \(L_{e,\Omega}\) of a surface is the flux of energy per unit solid angle, per unit projected area of the source.

\sphinxAtStartPar
\sphinxstylestrong{Spectral radiance in frequency} is the radiance per unit frequency, \(L_{e, \Omega, f} = \frac{\partial L_{e,\Omega}}{\partial f}\).
\end{sphinxadmonition}


\subsection{Fermi\sphinxhyphen{}Dirac}
\label{\detokenize{ch/statistical-mechanics/notes:fermi-dirac}}\label{\detokenize{ch/statistical-mechanics/notes:statistical-mechanics-notes-distributions-fd}}
\sphinxAtStartPar
Statistics of undistinguishable components that can’t be in the same (sub)level, obeying to the Pauli exclusion principle.
Given the number of elementary components \(\sum_{i} N_i = N\) and the energy \(\sum_{i} N_i E_i = E\),
\begin{equation}\label{equation:ch/statistical-mechanics/notes:eq:fd}
\begin{split}W_{FD,i} = \frac{G_i!}{(G_i-N_i)! N_i!} \qquad , \qquad W_{FD} = \prod_i W_{FD,i} \ .\end{split}
\end{equation}\subsubsection*{Counting microstates}

\sphinxAtStartPar
\sphinxstylestrong{todo} write page \sphinxstyleemphasis{Combinatorics} and add link

\sphinxAtStartPar
\sphinxstylestrong{Most likely microstate.} The approximate occupation number of the \(i^{th}\) level of the most likely microstate is
\begin{equation*}
\begin{split}n_i := \frac{N_i}{G_i} = \frac{1}{1 + e^{\alpha + \beta E_i}} \ .\end{split}
\end{equation*}\subsubsection*{Optimization}
\begin{equation*}
\begin{split}\begin{aligned}
  J(N_i, \alpha, \beta)
  & = \ln W_{FD} + \alpha \left( N - \sum_i N_i \right) + \beta \left(E - \sum_i N_i E_i \right) = \\
  & = \sum_i \left\{ \ln G_i! - \ln(G_i - N_i)! - \ln N_i!  \right\} + \alpha \left( N - \sum_i N_i \right) + \beta \left(E - \sum_i N_i E_i \right) = \\
  & = \sum_i \left\{ G_i \ln G_i - (G_i - N_i) \ln(G_i - N_i) - N_i \ln N_i \right\} + \alpha \left( N - \sum_i N_i \right) + \beta \left(E - \sum_i N_i E_i \right) = \\
\end{aligned}\end{split}
\end{equation*}
\sphinxAtStartPar
Using \(\partial_{n} (n+a) \ln (n+a) = \ln (n+a) + 1\),
\begin{equation*}
\begin{split}\begin{aligned}
  0 = \partial_{N_k} J 
  & \simeq \left\{ \ln (G_k - N_k) - \ln N_k \right\} - \alpha - \beta E_k \ ,
\end{aligned}\end{split}
\end{equation*}
\sphinxAtStartPar
and thus
\begin{equation*}
\begin{split}\ln \frac{G_k - N_k}{N_k} = \alpha + \beta E_k \ ,\end{split}
\end{equation*}\begin{equation*}
\begin{split}\frac{G_k}{N_k} - 1 = e^{\alpha + \beta E_k} \ \end{split}
\end{equation*}
\sphinxAtStartPar
The occupation number of the \(k\) level is
\begin{equation*}
\begin{split}N_k = \frac{G_k}{1 + e^{\alpha + \beta E_k}} \ .\end{split}
\end{equation*}
\sphinxAtStartPar
The average occupation of the \(G_k\) sublevels of the \(k\) level is
\begin{equation*}
\begin{split}n_k := \frac{N_k}{G_k} = \frac{1}{1 + e^{\alpha + \beta E_k}} \ .\end{split}
\end{equation*}\subsubsection*{\sphinxstylestrong{Meaning of \protect\(\alpha\protect\), \protect\(\beta\protect\)}}


\bigskip\hrule\bigskip


\sphinxstepscope


\chapter{Statistical Physics \sphinxhyphen{} Statistics Miscellanea}
\label{\detokenize{ch/statistical-mechanics/statistics:statistical-physics-statistics-miscellanea}}\label{\detokenize{ch/statistical-mechanics/statistics:statistical-mechanics-statistics}}\label{\detokenize{ch/statistical-mechanics/statistics::doc}}\subsubsection*{Information content and Entropy}

\sphinxAtStartPar
Given a discrete random variable \(X\) with probability mass function \(p_X(x)\), the self\sphinxhyphen{}information (\sphinxstylestrong{todo} \sphinxstyleemphasis{what about mutual information of random variables?}) is defined as the opposite of the logaritm of the mass function \(p_X(x)\),
\begin{equation*}
\begin{split}I_X(x) := - \ln \left( p_X(x) \right) \ .\end{split}
\end{equation*}
\sphinxAtStartPar
Information content of indenpendent random variables is additive. Since \(p_{X,Y}(x,y) = p_X(x) p_Y(y)\),
\begin{equation*}
\begin{split}I_{X,Y}(x,y) = - \ln \left( p_{X,Y}(x,y) \right) = -\ln \left( p_X(x) p_Y(y) \right) = - \ln p_X(x) - \ln p_Y(y) \ .\end{split}
\end{equation*}
\sphinxAtStartPar
\sphinxstylestrong{Shannon entropy.} Shannon entropy of a discrete random variable \(X\) is defined as the expected value of the information content,
\begin{equation*}
\begin{split}H(X) := \mathbb{E}[ I_X(X)] = \sum p_X(x) I_X(x) = - \sum p_X \ln p_X(x) \ .\end{split}
\end{equation*}
\sphinxAtStartPar
\sphinxstylestrong{Gibbs entropy.} Gibbs entropy was defined by J.W.Gibbs in 1878,
\begin{equation*}
\begin{split}S = - k_B \sum_i p_i \ln p_i \ .\end{split}
\end{equation*}
\sphinxAtStartPar
Additivity holds for independent random variables.

\sphinxAtStartPar
\sphinxstylestrong{Boltzmann entropy.} Boltmann entropy holds for uniform distributions over \(\Omega\) possible states, \(p_i = \frac{1}{\Omega}\). Gibbs’ entropy of this uniform distribution becomes
\begin{equation*}
\begin{split}S = - k_B \Omega \frac{1}{\Omega} \ln \frac{1}{\Omega} = k_B \ln \Omega \ .\end{split}
\end{equation*}
\sphinxAtStartPar
\sphinxstylestrong{Entropy in Quantum Mechanics.} \sphinxstylestrong{todo}
\subsubsection*{Boltzmann distribution}

\sphinxAtStartPar
Given a set of discrete states with probability \(p_i\), and the average measure as “macroscopic quantity” \(E = \sum_i p_i E_i\), Boltzann distribution maximizes the entropy (\sphinxstylestrong{todo} \sphinxstyleemphasis{Link to min info, max uncertainty})
\begin{equation*}
\begin{split}S = - k_B \sum_i p_i \ln p_i \ .\end{split}
\end{equation*}
\sphinxAtStartPar
The distribution follows from the constrained optimization
\begin{equation*}
\begin{split}\widetilde{S} = S - \alpha \left( \sum_i p_i -1  \right) - \beta \left( \sum_i p_i E_i - E \right)\end{split}
\end{equation*}\begin{equation*}
\begin{split}\begin{aligned}
  0 & = \partial_{\alpha} \widetilde{S} = - \sum_i p_i - 1 \\
  0 & = \partial_{\beta}  \widetilde{S} = - \sum_i p_i E_i - E \\
  0 & = \partial_{p_k}    \widetilde{S} = -k_B \left( \ln p_k + 1 \right) - \alpha - \beta E_k \\
\end{aligned}\end{split}
\end{equation*}
\sphinxAtStartPar
and thus
\begin{equation*}
\begin{split}p_k = e^{- 1 - \frac{\alpha}{k_B} - \frac{\beta}{k_B} E_k} = e^{-\left( 1 + \frac{\alpha}{k_B} \right)} e^{- \frac{\beta}{k_B} E_k} = C e^{-\frac{\beta}{k_B} E_k} \ ,\end{split}
\end{equation*}
\sphinxAtStartPar
and the normalization constant \(C\) is determined by normalization condition
\begin{equation*}
\begin{split}1 = \sum_k p_k = C \sum_{k} e^{-\frac{\beta E_k}{k_B}}\end{split}
\end{equation*}
\sphinxAtStartPar
The inverse \(Z = C^{-1}\) is defined as the \sphinxstylestrong{partition function},
\begin{equation*}
\begin{split} Z = C^{-1} = \sum_k e^{-\frac{\beta E_k}{k_B}}\ ,\end{split}
\end{equation*}
\sphinxAtStartPar
and the probability distribution becomes
\begin{equation*}
\begin{split}p_k = \frac{e^{-\frac{\beta E_k}{k_B}}}{Z} =  \frac{e^{-\frac{\beta E_k}{k_B}}}{ \sum_{i} e^{-\frac{\beta E_i}{k_B}}} \ .\end{split}
\end{equation*}
\sphinxAtStartPar
\sphinxstylestrong{Properties.}
\begin{equation*}
\begin{split}\frac{p_k}{p_i} = e^{-\frac{\beta}{k_B}(E_k - E_i)} \ .\end{split}
\end{equation*}\subsubsection*{Thermodynamics. Comparison of statistics and classical thermodynamics}

\sphinxAtStartPar
First principle of classical thermodynamics (for a monocomponent gas with no electric charge,…) reads
\begin{equation*}
\begin{split}T \, dS = d E + P \, dV\end{split}
\end{equation*}
\sphinxAtStartPar
\sphinxstylestrong{Entropy for Boltzmann distribution} reads
\begin{equation*}
\begin{split}\begin{aligned}
  S 
  & = - k_B \sum_i p_i \ln p_i = \\
  & = - k_B \sum_i \left[ p_i \left( - \frac{\beta E_i}{k_B} - \ln Z \right) \right] = \\
  & = \beta \langle E \rangle + k_B \ln Z
\end{aligned}\end{split}
\end{equation*}
\sphinxAtStartPar
From classical thermodyamics, temperature \(T\) can be defined as the partial derivative of the entropy of a system w.r.t. its internal energy keeping constant all the other independent variables,
\begin{equation*}
\begin{split}\begin{aligned}
  \frac{1}{T}
  & = \left.\left( \dfrac{\partial S}{\partial E} \right)\right|_X = \\
  & = \dfrac{\partial \beta}{\partial E} E + \beta + k_B \frac{\partial \ln Z}{\partial E} = \\
  & = \dfrac{\partial \beta}{\partial E} E + \beta + k_B \frac{1}{Z} \frac{\partial Z}{\partial E} = \\
  & = \dfrac{\partial \beta}{\partial E} E + \beta + k_B \frac{1}{Z} \frac{\partial Z}{\partial \beta} \frac{\partial \beta}{\partial E} = \\
  & = \dfrac{\partial \beta}{\partial E} E + \beta + k_B \frac{1}{Z} \left( - \sum_i \frac{E_i}{k_B} e^{-\frac{\beta E_i}{k_B}} \right) \frac{\partial \beta}{\partial E} = \\
  & = \dfrac{\partial \beta}{\partial E} E + \beta - \left( \sum_i E_i p_i \right) \frac{\partial \beta}{\partial E} = \\
  & = \dfrac{\partial \beta}{\partial E} E + \beta - E \frac{\partial \beta}{\partial E} = \beta \ .
\end{aligned}\end{split}
\end{equation*}
\sphinxAtStartPar
\sphinxstylestrong{todo}
\begin{itemize}
\item {} 
\sphinxAtStartPar
write the derivative above clearly in terms of composite functions

\item {} 
\sphinxAtStartPar
microscopical/statistical approach to the first principle of thermodynamics

\end{itemize}
\begin{equation*}
\begin{split}d E = d \left( \sum_i p_i E_i \right) = \sum_i E_i \, d p_i + \sum_i p_i d E_i\end{split}
\end{equation*}
\sphinxstepscope


\part{Quantum Mechanics}

\sphinxstepscope


\chapter{Quantum Mechanics}
\label{\detokenize{ch/quantum-mechanics/intro:quantum-mechanics}}\label{\detokenize{ch/quantum-mechanics/intro:quantum-mechanics-intro}}\label{\detokenize{ch/quantum-mechanics/intro::doc}}\begin{itemize}
\item {} 
\sphinxAtStartPar
Principles and postulates
\begin{itemize}
\item {} 
\sphinxAtStartPar
statistics and measurements outcomes (Heisenberg built its matrix mechanics only on observables…)

\item {} 
\sphinxAtStartPar
CCR

\end{itemize}

\item {} 
\sphinxAtStartPar
angluar momentum, spin, and atom

\end{itemize}


\section{Mathematical tools for quantum mechanics}
\label{\detokenize{ch/quantum-mechanics/intro:mathematical-tools-for-quantum-mechanics}}\label{ch/quantum-mechanics/intro:Operator}
\begin{sphinxadmonition}{note}{Definition 1 (Operator)}


\end{sphinxadmonition}
\label{ch/quantum-mechanics/intro:Adjoint Operator}
\begin{sphinxadmonition}{note}{Definition 2 (Adjoint operator)}



\sphinxAtStartPar
Given an operator \(\hat{A}: U \rightarrow V\), its self\sphinxhyphen{}adjoint \(\hat{A}^*: V \rightarrow U\) is the operator s.t.
\begin{equation*}
\begin{split}(\mathbf{v}, \ \hat{A} \mathbf{u})_{V} = (\mathbf{u}, \hat{A}^* \mathbf{v} )_{U}\end{split}
\end{equation*}
\sphinxAtStartPar
holds for \(\forall \mathbf{u} \in U, \ \mathbf{v} \in V\).
\end{sphinxadmonition}
\label{ch/quantum-mechanics/intro:Self-Adjoint Operator}
\begin{sphinxadmonition}{note}{Definition 3 (Hermitian (self\sphinxhyphen{}adjoint) operator)}



\sphinxAtStartPar
If \(\hat{A}: U \rightarrow U\), it is a self\sphinxhyphen{}adjoint operator if \(\hat{A}^* = \hat{A}\).
\end{sphinxadmonition}

\sphinxAtStartPar
Self\sphinxhyphen{}adjoint operators have real eigenvalues, and orthogonal eigenvectors (at least those associated to different eigenvalues; those associated with the same eigenvalues can be used to build an orthogonal set of vectors with orthogonalization process).


\section{Postulates of Quantum Mechanics}
\label{\detokenize{ch/quantum-mechanics/intro:postulates-of-quantum-mechanics}}\begin{itemize}
\item {} 
\sphinxAtStartPar
…

\item {} 
\sphinxAtStartPar
Canonical Commutation Relation (CCR) \sphinxstyleemphasis{and Canonical Anti\sphinxhyphen{}Commutation Relation…}

\item {} 
\sphinxAtStartPar
…

\end{itemize}


\section{Non\sphinxhyphen{}relativistic Mechanics}
\label{\detokenize{ch/quantum-mechanics/intro:non-relativistic-mechanics}}

\subsection{Statistical Interpretation and Measurement}
\label{\detokenize{ch/quantum-mechanics/intro:statistical-interpretation-and-measurement}}

\subsubsection{Wave function}
\label{\detokenize{ch/quantum-mechanics/intro:wave-function}}
\sphinxAtStartPar
The state of a system is described by a wave function \(|\Psi\rangle\)

\sphinxAtStartPar
\sphinxstylestrong{todo}
\begin{itemize}
\item {} 
\sphinxAtStartPar
properties: domain, image,…

\item {} 
\sphinxAtStartPar
unitary \(1 = \langle \Psi | \Psi \rangle = \left| \Psi \right|^2\), for statistical interpretation of \(\left| \Psi \right|^2\) as a density probability function

\end{itemize}


\subsubsection{Operators and Observables}
\label{\detokenize{ch/quantum-mechanics/intro:operators-and-observables}}
\sphinxAtStartPar
Physical \sphinxstylestrong{observable} quantities are represented by {\hyperref[\detokenize{ch/quantum-mechanics/intro:quantum-mechanics-math-operators-self-adjoint}]{\sphinxcrossref{\DUrole{std,std-ref}{Hermitian operators}}}}. Possible outcomes of measurement are the eigenvalues of the operator

\sphinxAtStartPar
Given \(\hat{A}\) and the set of its eigenvectors \(\{ |A_i \rangle \}_i\) (\sphinxstylestrong{todo} \sphinxstyleemphasis{continuous or discrete spectrum…, need to treat this difference quite in details}), with associated eigenvalues \(\{ a_i \}_i\)
\begin{equation*}
\begin{split}\hat{A} |A_i \rangle = a_i |A_i\rangle\end{split}
\end{equation*}\begin{equation*}
\begin{split}| \Psi \rangle = | A_i \rangle \langle A_i | \Psi \rangle =  | A_i \rangle \Psi_i^A \end{split}
\end{equation*}\begin{equation*}
\begin{split}\langle A_j | \Psi \rangle = \langle A_j | A_i \rangle \langle A_i | \Psi \rangle = \Psi_j^A \end{split}
\end{equation*}
\sphinxAtStartPar
and thus
\begin{equation*}
\begin{split}\begin{aligned}
  \Psi_j^A    & = \langle A_j  | \Psi \rangle \\
  \Psi_j^{A*} & = \langle \Psi | A_j \rangle \\
\end{aligned}\end{split}
\end{equation*}\begin{itemize}
\item {} 
\sphinxAtStartPar
identity operator \(\sum_i | A_i \rangle \langle A_i | = \mathbb{I}\), since
\begin{equation*}
\begin{split}\sum_i | A_i \rangle \langle A_i | \Psi \rangle = \sum_i | A_i \rangle \langle A_i | \Psi_j^A A_j \rangle =
  \sum_i | A_i \rangle \delta_{ij} \Psi_j^A = \sum_i | A_i \rangle  \Psi_i^A  = | \Psi \rangle\end{split}
\end{equation*}
\item {} 
\sphinxAtStartPar
Normalization:
\begin{equation*}
\begin{split}1 =\langle \Psi | \Psi \rangle = \Psi_j^{A*} \underbrace{\langle A_j | A_i \rangle}_{\delta_{ij}} \Psi_i^{A} = \sum_i \left| \Psi_i^A \right|^2\end{split}
\end{equation*}
\sphinxAtStartPar
with \(|\Psi_i^A|^2\) that can be interpreted as the probability of finding the system in state \(|\Psi_i^a\rangle\)

\item {} 
\sphinxAtStartPar
Expected value of the physical quantity in the a state \(|\Psi\rangle\), with possible values \(a_i\) with probability \(|\Psi_i^A|^2\)
\begin{equation*}
\begin{split}\begin{aligned}
    \bar{A}_{\Psi} & = \sum_i a_i |\Psi_i^A|^2 = \\
            & = \sum_i a_i \Psi_i^{A*} \Psi_i^A =  \\ 
            & = \sum_i a_i \langle \Psi | A_i \rangle \langle A_i | \Psi \rangle = \\
            & = \langle \Psi | \left( \sum_i a_i | A_i \rangle \langle A_i | \right) | \Psi \rangle = \\
            & = \langle \Psi | \hat{A} | \Psi \rangle = \\
  \end{aligned}\end{split}
\end{equation*}
\sphinxAtStartPar
since an operator \(\hat{A}\) can be written as a function of its eigenvalues and eigenvectors
\begin{equation*}
\begin{split}\begin{aligned}
   \left( \sum_i a_i | A_i \rangle \langle A_i | \right) \Psi \rangle 
   & = \left( \sum_i a_i | A_i \rangle \langle A_i | \right) c_k | A_k \rangle = \\
   & =  \sum_i a_i | A_i \rangle c_i = \\
   & =  \sum_i \hat{A} | A_i \rangle c_i = \\
   & = \hat{A} \sum_i | A_i \rangle c_i = \hat{A} | \Psi \rangle \ . 
   \end{aligned}\end{split}
\end{equation*}
\end{itemize}




\subsubsection{Space Representation}
\label{\detokenize{ch/quantum-mechanics/intro:space-representation}}
\sphinxAtStartPar
\sphinxstylestrong{Position operator} \(\hat{\mathbf{r}}\) has eigenvalues \(\mathbf{r}\) identifying the possible measurements of the position
\begin{equation*}
\begin{split}\hat{\mathbf{r}} | \mathbf{r} \rangle = \mathbf{r} | \mathbf{r} \rangle \ ,\end{split}
\end{equation*}
\sphinxAtStartPar
being \(\mathbf{r}\) the result of the measurement (position in space, mathematically it could be a vector), and \(| \mathbf{r} \rangle\) the state function corresponding to the measurement \(\mathbf{r}\) of the position.
\begin{itemize}
\item {} 
\sphinxAtStartPar
Result of measurement, \(\mathbf{r}\), is a position in space. As an example, it could be a point in an Euclidean space \(P \in E^n\). It could be written using properties of Dirac’s delta “function”
\begin{equation*}
\begin{split}
    \mathbf{r} = \int_{\mathbf{r}'} \delta (\mathbf{r}'-\mathbf{r}) \, \mathbf{r}' d \mathbf{r}'
  \end{split}
\end{equation*}
\item {} 
\sphinxAtStartPar
Projection of wave function over eigenstates of position operator
\begin{equation*}
\begin{split}\begin{aligned}
  \langle \mathbf{r} | \Psi \rangle(t) = \Psi(\mathbf{r}, t)
  & = \int_{\mathbf{r}'} \delta(\mathbf{r} - \mathbf{r}') \Psi(\mathbf{r}',t) d \mathbf{r}' = \\
  & = \int_{\mathbf{r}'} \langle \mathbf{r} | \mathbf{r}' \rangle \Psi(\mathbf{r}',t) d \mathbf{r}' = \\
  & = \int_{\mathbf{r}'} \langle \mathbf{r} | \mathbf{r}' \rangle \langle \mathbf{r}' | \Psi \rangle (t) d \mathbf{r}' = \\
  & = \langle \mathbf{r} | \underbrace{\left( \int_{\mathbf{r}'} | \mathbf{r}' \rangle \langle \mathbf{r}' | d \mathbf{r}' \right)}_{= \hat{\mathbf{1}}} | \Psi \rangle (t) \ .
  \end{aligned}\end{split}
\end{equation*}
\item {} 
\sphinxAtStartPar
having used orthogonality (\sphinxstylestrong{todo} \sphinxstyleemphasis{why? provide definition and examples of operators with continuous spectrum})
\begin{equation*}
\begin{split}\langle \mathbf{r}' | \mathbf{r} \rangle = \delta(\mathbf{r}' - \mathbf{r})\end{split}
\end{equation*}
\item {} 
\sphinxAtStartPar
Expansion of a state function \(|\Psi\rangle (t)\) over the basis of the position operator
\begin{equation*}
\begin{split}\begin{aligned}
  | \Psi \rangle (t) = \hat{\mathbf{1}} | \Psi \rangle(t) 
    = \left( \int_{\mathbf{r}'} | \mathbf{r}' \rangle \langle \mathbf{r}' d \mathbf{r}' \right) | \Psi \rangle(t) 
    = \int_{\mathbf{r}'} | \mathbf{r}' \rangle \langle \mathbf{r}' | \Psi \rangle(t) \, d \mathbf{r}' \ .
  \end{aligned}\end{split}
\end{equation*}
\item {} 
\sphinxAtStartPar
Unitariety and probability density
\begin{equation*}
\begin{split}\begin{aligned}
  1 = \langle \Psi | \Psi \rangle (t) 
    & = \langle \Psi | \left( \int_{\mathbf{r}'} | \mathbf{r}' \rangle \langle \mathbf{r}' d \mathbf{r}' \right) | \Psi \rangle \\
    & = \int_{\mathbf{r}'} \langle \Psi | \mathbf{r}' \rangle \langle \mathbf{r}' | \Psi \rangle \, d \mathbf{r}' \\
    & = \int_{\mathbf{r}'} \Psi^*(\mathbf{r}',t) \Psi(\mathbf{r}',t) \, d \mathbf{r}' \\
    & = \int_{\mathbf{r}'} \left| \Psi(\mathbf{r}',t) \right|^2 \, d \mathbf{r}' \\
  \end{aligned}\end{split}
\end{equation*}
\sphinxAtStartPar
and thus \(\left|\Psi(\mathbf{r},t)\right|^2\) can be interpreted as the \sphinxstylestrong{probability density function} of measuring position of the system equal to \(\mathbf{r}'\).

\item {} 
\sphinxAtStartPar
Average value of the operator
\begin{equation*}
\begin{split}\begin{aligned}
  \bar{\mathbf{r}} & = \langle \Psi | \hat{\mathbf{r}} | \Psi \rangle = \\
  & = \int_{\mathbf{r}'} \langle \Psi | \mathbf{r}' \rangle \langle \mathbf{r}' | d \mathbf{r}' \ | \hat{\mathbf{r}} | \ \int_{\mathbf{r}''} | \mathbf{r}'' \rangle \langle \mathbf{r}'' | \Psi \rangle \, d \mathbf{r}'' \\
  & = \int_{\mathbf{r}'} \int_{\mathbf{r}''} \langle \Psi | \mathbf{r}' \rangle \langle \mathbf{r}' | \hat{\mathbf{r}} |  \mathbf{r}'' \rangle \langle \mathbf{r}'' | \Psi \rangle \, d \mathbf{r}'  d \mathbf{r}'' = \\
  & = \int_{\mathbf{r}'} \int_{\mathbf{r}''} \langle \Psi | \mathbf{r}' \rangle \underbrace{\langle \mathbf{r}' |  \mathbf{r}'' \rangle}_{=\delta(\mathbf{r}'-\mathbf{r}'')} \mathbf{r}'' \langle \mathbf{r}'' | \Psi \rangle \, d \mathbf{r}'  d \mathbf{r}'' = \\
  & = \int_{\mathbf{r}'} \langle \Psi | \mathbf{r}' \rangle \mathbf{r}' \langle \mathbf{r}' | \Psi \rangle \, d \mathbf{r}' = \\
  & = \int_{\mathbf{r}'} \Psi^*(\mathbf{r}',t) \ \mathbf{r}' \ \Psi(\mathbf{r}',t) \, d \mathbf{r}' = \\
  & = \int_{\mathbf{r}'}  \left| \Psi(\mathbf{r}',t) \right|^2 \, \mathbf{r}' \, d \mathbf{r}' \ .
  \end{aligned}\end{split}
\end{equation*}
\end{itemize}


\subsubsection{Momentum Representation}
\label{\detokenize{ch/quantum-mechanics/intro:momentum-representation}}
\sphinxAtStartPar
\sphinxstylestrong{Momentum operator} as the limit of…\sphinxstylestrong{todo} \sphinxstyleemphasis{prove the expression of the momentum operator as the limit of the generator of translation}
\begin{equation*}
\begin{split}\langle \mathbf{r} | \hat{\mathbf{p}} = - i \hbar \nabla \langle \mathbf{r} | \end{split}
\end{equation*}\begin{itemize}
\item {} 
\sphinxAtStartPar
Spectrum
\begin{equation*}
\begin{split}\hat{\mathbf{p}} | \mathbf{p} \rangle = \mathbf{p} | \mathbf{p} \rangle\end{split}
\end{equation*}\begin{equation*}
\begin{split}\langle \mathbf{r} | \hat{\mathbf{p}} | \mathbf{p} \rangle = - i \hbar \nabla \langle \mathbf{r} | \mathbf{p} \rangle = \mathbf{p} \langle \mathbf{r} | \mathbf{p} \rangle\end{split}
\end{equation*}
\sphinxAtStartPar
and thus the eigenvectors in space base \(\mathbf{p}(\mathbf{r}) = \langle \mathbf{r} | \mathbf{p} \rangle\) are the solution of the differential equation
\begin{equation*}
\begin{split}- i \hbar \nabla \mathbf{p}(\mathbf{r}) = \mathbf{p} \mathbf{p}(\mathbf{r}) \ ,\end{split}
\end{equation*}
\sphinxAtStartPar
that in Cartesian coordinates reads
\begin{equation*}
\begin{split}- i \hbar \partial_j p_k(\mathbf{r}) = p_j p_k(\mathbf{r})\end{split}
\end{equation*}\begin{equation*}
\begin{split}p_k(\mathbf{r}) = p_{k,0} \exp \left[ i \frac{p_j}{\hbar} r_j \right]\end{split}
\end{equation*}
\sphinxAtStartPar
or
\begin{equation*}
\begin{split}\langle \mathbf{r} | \mathbf{p} \rangle = \mathbf{p}(\mathbf{r}) = \mathbf{p}_0 \exp \left[ i \frac{\mathbf{p} \cdot \mathbf{r}}{\hbar} \right]\end{split}
\end{equation*}
\sphinxAtStartPar
\sphinxstylestrong{todo}
\begin{itemize}
\item {} 
\sphinxAtStartPar
normalization factor \(\frac{1}{(2 \pi)^{\frac{3}{2}}}\)
\begin{equation*}
\begin{split}\mathscr{F}\{ \delta(x) \}(k) = \int_{-\infty}^{\infty} \delta(x) e^{-ikx} \, dx = 1\end{split}
\end{equation*}
\item {} 
\sphinxAtStartPar
Fourier transform and inverse Fourier transform: definitions and proofs (link to a math section)

\item {} 
\sphinxAtStartPar
representation in basis of wave vector operator \(\hat{\mathbf{k}}\), \(\hat{\mathbf{p}} = \hbar \hat{\mathbf{k}}\)

\end{itemize}

\end{itemize}


\subsubsection{From position to momentum representation}
\label{\detokenize{ch/quantum-mechanics/intro:from-position-to-momentum-representation}}
\sphinxAtStartPar
Momentum and wave vector, \(\mathbf{p} = \hbar \mathbf{k}\)
\begin{equation*}
\begin{split}\begin{aligned}
  \langle \mathbf{p} | \Psi \rangle
  & = \langle \mathbf{p} | \int_{\mathbf{r}'} | \mathbf{r}' \rangle \langle \mathbf{r}' | \Psi\rangle d \mathbf{r}' = \\
  & =   \int_{\mathbf{r}'} \langle \mathbf{p} | \mathbf{r}' \rangle \langle \mathbf{r}' | \Psi\rangle d \mathbf{r}' = \\
  & = \frac{1}{(2\pi)^{3/2}} \int_{\mathbf{r}'} \exp \left[ i \frac{\mathbf{p} \cdot \mathbf{r}}{\hbar} \right] \langle \mathbf{r}' | \Psi\rangle d \mathbf{r}' = \\
\end{aligned}\end{split}
\end{equation*}
\sphinxAtStartPar
Relation between position and wave\sphinxhyphen{}number representation can be represented with a Fourier transform
\begin{equation*}
\begin{split}\begin{aligned}
  \langle \mathbf{k} | \Psi \rangle
  & = \langle \mathbf{k} | \int_{\mathbf{r}'} | \mathbf{r}' \rangle \langle \mathbf{r}' | \Psi\rangle d \mathbf{r}' = \\
  & =   \int_{\mathbf{r}'} \langle \mathbf{k} | \mathbf{r}' \rangle \langle \mathbf{r}' | \Psi\rangle d \mathbf{r}' = \\
  & = \frac{1}{(2\pi)^{3/2}} \int_{\mathbf{r}'} \exp \left[ i \mathbf{k} \cdot \mathbf{r}' \right] \langle \mathbf{r}' | \Psi\rangle d \mathbf{r}' = \\
  & = \frac{1}{(2\pi)^{3/2}} \int_{\mathbf{r}'} \exp \left[ i \mathbf{k} \cdot \mathbf{r}' \right] \Psi(\mathbf{r}') d \mathbf{r}' = \\
  & = \mathscr{F}\{ \Psi(\mathbf{r}) \} (\mathbf{k})
\end{aligned}\end{split}
\end{equation*}

\subsection{Schrodinger Equation}
\label{\detokenize{ch/quantum-mechanics/intro:schrodinger-equation}}\begin{equation*}
\begin{split}i \hbar \dfrac{d}{dt} | \Psi \rangle = \hat{H} | \Psi \rangle \end{split}
\end{equation*}
\sphinxAtStartPar
being \(\hat{H}\) the Hamiltonian operator and \(|\Psi\rangle\) the wave function, as a function of time \(t\) as an independent variable.


\subsubsection{Stationary States}
\label{\detokenize{ch/quantum-mechanics/intro:stationary-states}}
\sphinxAtStartPar
Eigenspace of the Hamiltonian operator
\begin{equation*}
\begin{split}\hat{H} |\Psi_k \rangle = E_k |\Psi_k \rangle \ ,\end{split}
\end{equation*}
\sphinxAtStartPar
with \(E_k\) possible values of energy measurements. \sphinxstyleemphasis{If no eigenstates with the same eigenvalue exists, then…otherwise…}
\sphinxstyleemphasis{Without external influence} \sphinxstylestrong{todo} \sphinxstyleemphasis{be more detailed!}, energy values and eigenstates of the systems are constant in time.

\sphinxAtStartPar
Thus, exapnding the state of the system \(|\Psi\rangle\) over the stationary states gives \(|\Psi_k\rangle\), \(|\Psi\rangle = |\Psi_k\rangle c_k(t)\), and inserting in Schrodinger equation
\begin{equation*}
\begin{split}i \hbar \dot{c}_k |\Psi_k\rangle = c_k  E_k |\Psi_k\rangle\end{split}
\end{equation*}
\sphinxAtStartPar
and exploiting orthogonality of eigenstates, a diagonal system for the amplitudes of stationary states ariese,
\begin{equation*}
\begin{split}i \hbar \dot{c}_k = c_k E_k \ .\end{split}
\end{equation*}
\sphinxAtStartPar
whose solution reads
\begin{equation*}
\begin{split}c_k(t) = c_{k,0} \exp \left[- i \frac{E_k}{\hbar} t \right]\end{split}
\end{equation*}
\sphinxAtStartPar
Thus the state of the system evolves like a superposition of monochromatic waves with frequencies \(\omega_k = \frac{E_k}{\hbar}\),
\begin{equation*}
\begin{split}| \Psi \rangle = | \Psi_k \rangle c_k(t) = | \Psi_k \rangle c_{k,0}\exp\left[-i \frac{E_k}{\hbar} t \right] \ .\end{split}
\end{equation*}\begin{equation*}
\begin{split}\begin{aligned}
 \dfrac{d}{dt} \bar{A}
 & = \dfrac{d}{dt} \left( \langle \Psi | \hat{A} | \Psi \rangle \right) = \\
 & = \dfrac{d}{dt} \langle \Psi | \hat{A} | \Psi \rangle + \langle \Psi | \frac{d \hat{A}}{dt} | \Psi \rangle + \langle \Psi | \hat{A} \frac{d}{dt} | \Psi \rangle = \\
 & = \langle \Psi | \frac{d \hat{A}}{dt} | \Psi \rangle + \frac{i}{\hbar} \langle \Psi | \hat{H} \hat{A} | \Psi \rangle - \frac{i}{\hbar} \langle \Psi | \hat{A} \hat{H} | \Psi \rangle = \\
 & = \langle \Psi | \left( \frac{i}{\hbar} [ \hat{H}, \hat{A} ] + \frac{d \hat{A}}{dt} \right) | \Psi \rangle \ .
\end{aligned}\end{split}
\end{equation*}

\subsubsection{Pictures}
\label{\detokenize{ch/quantum-mechanics/intro:pictures}}\begin{itemize}
\item {} 
\sphinxAtStartPar
Schrodinger

\item {} 
\sphinxAtStartPar
Heisenberg

\item {} 
\sphinxAtStartPar
Interaction

\end{itemize}


\paragraph{Schrodinger}
\label{\detokenize{ch/quantum-mechanics/intro:schrodinger}}
\sphinxAtStartPar
If \(\hat{H}\) not function of time
\begin{equation*}
\begin{split}| \Psi \rangle (t) = \exp\left[ - i \frac{\hat{H}}{\hbar} (t-t_0) \right] | \Psi \rangle(t_0) = \hat{U}(t,t_0) | \Psi \rangle(t_0) \end{split}
\end{equation*}\begin{equation*}
\begin{split}\bar{A} = \langle \Psi | \hat{A} | \Psi \rangle = \langle \Psi_0 | \hat{U}^*(t,t_0) \hat{A} \hat{U}(t,t_0) | \Psi_0 \rangle\end{split}
\end{equation*}

\paragraph{Heisenberg}
\label{\detokenize{ch/quantum-mechanics/intro:heisenberg}}
\sphinxAtStartPar
…

\sphinxAtStartPar
for \(\hat{H}\) independent from time \(t\),
\begin{equation*}
\begin{split}\begin{aligned}
  \dfrac{d}{dt} \bar{\mathbf{r}} & = \overline{\frac{i}{\hbar} \left[ \hat{H}, \hat{\mathbf{r}} \right]} \\
  \dfrac{d}{dt} \bar{\mathbf{p}} & = \overline{\frac{i}{\hbar} \left[ \hat{H}, \hat{\mathbf{p}} \right]} \\
\end{aligned}\end{split}
\end{equation*}\subsubsection*{Hamiltonian Mechanics}

\sphinxAtStartPar
From Lagrange equations
\begin{equation*}
\begin{split}\dfrac{d}{dt} \left( \frac{\partial L}{\partial \dot{q}} \right) - \frac{\partial L}{\partial q} = Q_q\end{split}
\end{equation*}
\sphinxAtStartPar
\(q\) generalized coordinates, \(p:= \frac{\partial L}{\partial \dot{q}}\) generalized momenta.

\sphinxAtStartPar
Hamiltonian
\begin{equation*}
\begin{split}H(p,q,t) = p \dot{q} - L(\dot{q}, q, t)\end{split}
\end{equation*}
\sphinxAtStartPar
Increment of the Hamiltonian
\begin{equation*}
\begin{split}dH = \partial_p H dp + \partial_q H dq + \partial_t H dt \end{split}
\end{equation*}\begin{equation*}
\begin{split}\begin{aligned}
  dH & = \dot{q} dp + p d \dot{q} - \partial_{\dot{q}} L d \dot{q} - \partial_{q} L d q - \partial_t L d t = \\
     & = \dot{q} dp - \partial_{q} L d q - \partial_t L d t = \\
     & = \dot{q} dp - \left( \dot{p} + Q_q \right) d q - \partial_t L d t = \\
\end{aligned}\end{split}
\end{equation*}\begin{equation*}
\begin{split}\begin{cases}
  \frac{\partial H}{\partial p} = \dot{q} \\
  \frac{\partial H}{\partial q} = - \frac{\partial L}{\partial q} = - \dot{p} + Q_q \\
  \frac{\partial H}{\partial t} = - \frac{\partial L}{\partial t}
\end{cases}\end{split}
\end{equation*}
\sphinxAtStartPar
Physical quantity \(f(p(t), q(t), t)\). Its time derivative reads
\begin{equation*}
\begin{split}\begin{aligned}
\frac{d f}{dt}
  & = \frac{\partial f}{\partial p} \dot{p} + \frac{\partial f}{\partial q} \dot{q} + \frac{\partial f}{\partial t} = \\
  & = \frac{\partial f}{\partial p} \left[ - \frac{\partial H}{\partial q} + Q_q \right] + \frac{\partial f}{\partial q} \frac{\partial H}{\partial p} + \frac{\partial f}{\partial t} = \\
  & = \{ H, f \} + \partial_t f + Q_q \partial_p f
\end{aligned}\end{split}
\end{equation*}
\sphinxAtStartPar
If \(Q_q = 0\), the correspondence between quantum mechanics and classical mechanics
\begin{equation*}
\begin{split}\frac{d f}{d t} = \{ H, f \} + \partial_t f \qquad \leftrightarrow \qquad \dfrac{d}{dt} \overline{\hat{f}} = \overline{\frac{i}{\hbar} [ \hat{H}, \hat{f} ]} + \overline{\frac{\partial \hat{f}}{\partial t}}\end{split}
\end{equation*}\begin{equation*}
\begin{split}\{ H, f \} \qquad \leftrightarrow \qquad \frac{i}{\hbar}[\hat{H}, \hat{f}]\end{split}
\end{equation*}

\paragraph{Interaction}
\label{\detokenize{ch/quantum-mechanics/intro:interaction}}

\subsection{Matrix Mechanics}
\label{\detokenize{ch/quantum-mechanics/intro:matrix-mechanics}}

\subsubsection{Attualization of 1925 papers}
\label{\detokenize{ch/quantum-mechanics/intro:attualization-of-1925-papers}}


\sphinxAtStartPar
…to find the canonical commutation relation,
\begin{equation*}
\begin{split}[ \hat{\mathbf{r}}, \hat{\mathbf{p}} ] = i \hbar  \mathbb{I} \, \hat{\mathbf{1}} \ .\end{split}
\end{equation*}\begin{equation*}
\begin{split}\begin{aligned}
\left[ \hat{\mathbf{r}}, \hat{\mathbf{p}} \right]
 & = \hat{\mathbf{r}} \hat{\mathbf{p}} - \hat{\mathbf{p}} \hat{\mathbf{r}} = \\
 & = \hat{\mathbf{r}} \int_{\mathbf{r}} | \mathbf{r} \rangle \langle \mathbf{r} | d \mathbf{r} \hat{\mathbf{p}}
   - \hat{\mathbf{p}} \int_{\mathbf{r}} | \mathbf{r} \rangle \langle \mathbf{r} | d \mathbf{r} \ \hat{\mathbf{r}} \ \int_{\mathbf{r}'} | \mathbf{r}' \rangle \langle \mathbf{r}' | d \mathbf{r}' = \\
 & = - \int_{\mathbf{r}} \mathbf{r} | \mathbf{r} \rangle i \hbar \nabla \langle \mathbf{r} | \, d \mathbf{r} - \hat{\mathbf{p}} \int_{\mathbf{r}} \int_{\mathbf{r}'} | \mathbf{r} \rangle \mathbf{r}' \underbrace{\langle \mathbf{r} |\mathbf{r}' \rangle}_{\delta(\mathbf{r}-\mathbf{r}')} \langle \mathbf{r}' | d \mathbf{r}' = \\
 & = - \int_{\mathbf{r}} \mathbf{r} | \mathbf{r} \rangle i \hbar \nabla \langle \mathbf{r} | \, d \mathbf{r} - \hat{\mathbf{p}} \int_{\mathbf{r}} \mathbf{r} | \mathbf{r} \rangle \langle \mathbf{r} | d \mathbf{r} = \\
 & = - \int_{\mathbf{r}} \mathbf{r} | \mathbf{r} \rangle i \hbar \nabla \langle \mathbf{r} | \, d \mathbf{r} - \int_{\mathbf{r}'} | \mathbf{r} \rangle \langle \mathbf{r} | d \mathbf{r} \ \hat{\mathbf{p}} \ \int_{\mathbf{r}'} \mathbf{r}' | \mathbf{r}' \rangle \langle \mathbf{r}' | d \mathbf{r}' = \\
 & = - \int_{\mathbf{r}} \mathbf{r} | \mathbf{r} \rangle i \hbar \nabla \langle \mathbf{r} | \, d \mathbf{r} + \int_{\mathbf{r}} | \mathbf{r} \rangle i \hbar \nabla \langle \mathbf{r} | d \mathbf{r} \int_{\mathbf{r}'} \mathbf{r}' | \mathbf{r}' \rangle \langle \mathbf{r}' | d \mathbf{r}' = \dots
\end{aligned}\end{split}
\end{equation*}\begin{equation*}
\begin{split}\begin{aligned}
\left[ \hat{\mathbf{r}}, \hat{\mathbf{p}} \right] | \Psi \rangle 
 & = - \int_{\mathbf{r}} \mathbf{r} | \mathbf{r} \rangle i \hbar \nabla \Psi(\mathbf{r},t) + \int_{\mathbf{r}} | \mathbf{r} \rangle i \hbar \nabla \left( \mathbf{r} \Psi(\mathbf{r},t) \right) = \\
 & = - \int_{\mathbf{r}} | \mathbf{r} \rangle i \hbar \left[ \mathbf{r} \nabla \Psi(\mathbf{r},t) + \mathbb{I} \Psi(\mathbf{r},t) + \mathbf{r} \nabla \Psi(\mathbf{r},t) \right] = \\
 & = i \hbar \underbrace{\int_{\mathbf{r}} | \mathbf{r} \rangle \langle \mathbf{r} | d \mathbf{r}}_{\hat{\mathbf{1}}} \, | \Psi \rangle \ ,
\end{aligned}\end{split}
\end{equation*}
\sphinxAtStartPar
and since \(| \Psi \rangle\) is arbitrary
\begin{equation*}
\begin{split}\left[ \hat{\mathbf{r}}, \hat{\mathbf{p}} \right] = i \hbar \mathbb{I} \hat{\mathbf{1}} \ . \end{split}
\end{equation*}\begin{equation*}
\begin{split}\left[ \hat{r}_a, \hat{p}_b \right] = i \hbar \delta_{ab} \ . \end{split}
\end{equation*}

\subsection{Heisenberg Uncertainty relation}
\label{\detokenize{ch/quantum-mechanics/intro:heisenberg-uncertainty-relation}}
\sphinxAtStartPar
Uncertainty principle is a relation that holds for “wave descriptions” as it can be proved in the generic framework of \sphinxhref{https://basics2022.github.io/bbooks-math-miscellanea/ch/complex/fourier-transform.html}{Fourier transform}, see \sphinxhref{https://basics2022.github.io/bbooks-math-miscellanea/ch/complex/fourier-transform.html\#uncertainty-relation}{Fourier transform:Uncertainty relation}.
\begin{itemize}
\item {} 
\sphinxAtStartPar
Heisenberg uncertainty relation is a relation between product of the variance of two physical quantities and their commutator,

\item {} 
\sphinxAtStartPar
\sphinxstylestrong{todo} \sphinxstyleemphasis{relation with measurement process and outcomes. Commutation as a measurement process: first measure \(B\) and then \(A\), or first measure \(A\) and then \(B\)}

\end{itemize}
\begin{equation*}
\begin{split}\sigma_A \sigma_B \ge \frac{1}{2} \left|\overline{[\hat{A}, \hat{B}]}\right| \ .\end{split}
\end{equation*}\subsubsection*{Proof of Heisenberg uncertainty “principle”}
\begin{equation*}
\begin{split}\begin{aligned}
 \sigma^2_A \sigma^2_B
 & = \langle \Psi | \left(\hat{A} - \bar{A} \right)^2 | \Psi \rangle\langle \Psi | \left(\hat{B} - \bar{B} \right)^2 | \Psi \rangle = \\
 & = \langle (\hat{A} - \bar{A} ) \Psi |  (\hat{A} - \bar{A} ) \Psi \rangle \langle  (\hat{B} - \bar{B} ) \Psi |  (\hat{B} - \bar{B} ) \Psi \rangle = \\
 & = \| ( \hat{A} - \bar{A} ) \Psi \|^2 \| ( \hat{B} - \bar{B} ) \Psi \|^2 = \\
 & \ge \left| \langle ( \hat{A} - \bar{A} ) \Psi | ( \hat{B} - \bar{B} ) \Psi  \rangle \right|^2 = \\
 & = \left| \langle \Psi | (\hat{A} - \bar{A})(\hat{B} - \bar{B}) \Psi \rangle \right|^2 = \\
 & = \left| \langle \Psi | \hat{A}\hat{B} - \hat{A}\bar{B} - \bar{A}\hat{B} + \bar{A}\bar{B} | \Psi \rangle \right|^2 = \\
 & = \left| \langle \Psi | \hat{A}\hat{B} - \bar{A}\bar{B} | \Psi \rangle \right|^2 \ge && \text{(1)} \\
 & = \left| \frac{\langle \Psi | \hat{A}\hat{B} - \hat{B}\hat{A} | \Psi \rangle}{2i} \right|^2 = \\
 & = \frac{ \left|\langle \Psi | [\hat{A}, \hat{B}] | \Psi \rangle \right|^2}{4} = \frac{1}{4} \left| \overline{[\hat{A}, \hat{B}]} \right|^2
\end{aligned}\end{split}
\end{equation*}
\sphinxAtStartPar
having used Cauchy\sphinxhyphen{}Schwartz triangle inequality in (1),
\begin{equation*}
\begin{split}|z| \ge | \text{im}(z) | = \frac{z - z^*}{2 i} \ .\end{split}
\end{equation*}


\sphinxAtStartPar
Hesienberg uncertainty principles applied to position and momentum reads
\begin{equation*}
\begin{split}\sigma_{r_a} \sigma_{p_b} \ge \frac{1}{2} \left|\overline{[\hat{r}_a, \hat{p}_b]}\right| = \frac{\hbar}{2}  \delta_{ab} \ .\end{split}
\end{equation*}

\section{Many\sphinxhyphen{}body problem}
\label{\detokenize{ch/quantum-mechanics/intro:many-body-problem}}
\sphinxAtStartPar
Wave function with symmetries: Fermions and Bosons

\sphinxstepscope


\chapter{Quantum Mechanics \sphinxhyphen{} Notes}
\label{\detokenize{ch/quantum-mechanics/notes:quantum-mechanics-notes}}\label{\detokenize{ch/quantum-mechanics/notes:id1}}\label{\detokenize{ch/quantum-mechanics/notes::doc}}





\renewcommand{\indexname}{Proof Index}
\begin{sphinxtheindex}
\let\bigletter\sphinxstyleindexlettergroup
\bigletter{Adjoint Operator}
\item\relax\sphinxstyleindexentry{Adjoint Operator}\sphinxstyleindexextra{ch/quantum\sphinxhyphen{}mechanics/intro}\sphinxstyleindexpageref{ch/quantum-mechanics/intro:\detokenize{Adjoint Operator}}
\indexspace
\bigletter{Operator}
\item\relax\sphinxstyleindexentry{Operator}\sphinxstyleindexextra{ch/quantum\sphinxhyphen{}mechanics/intro}\sphinxstyleindexpageref{ch/quantum-mechanics/intro:\detokenize{Operator}}
\indexspace
\bigletter{Self\sphinxhyphen{}Adjoint Operator}
\item\relax\sphinxstyleindexentry{Self\sphinxhyphen{}Adjoint Operator}\sphinxstyleindexextra{ch/quantum\sphinxhyphen{}mechanics/intro}\sphinxstyleindexpageref{ch/quantum-mechanics/intro:\detokenize{Self-Adjoint Operator}}
\indexspace
\bigletter{example\sphinxhyphen{}0}
\item\relax\sphinxstyleindexentry{example\sphinxhyphen{}0}\sphinxstyleindexextra{ch/statistical\sphinxhyphen{}mechanics/notes}\sphinxstyleindexpageref{ch/statistical-mechanics/notes:\detokenize{example-0}}
\indexspace
\bigletter{example\sphinxhyphen{}1}
\item\relax\sphinxstyleindexentry{example\sphinxhyphen{}1}\sphinxstyleindexextra{ch/statistical\sphinxhyphen{}mechanics/notes}\sphinxstyleindexpageref{ch/statistical-mechanics/notes:\detokenize{example-1}}
\end{sphinxtheindex}

\renewcommand{\indexname}{Index}
\printindex
\end{document}