%% Generated by Sphinx.
\def\sphinxdocclass{jupyterBook}
\documentclass[letterpaper,10pt,italian]{jupyterBook}
\ifdefined\pdfpxdimen
   \let\sphinxpxdimen\pdfpxdimen\else\newdimen\sphinxpxdimen
\fi \sphinxpxdimen=.75bp\relax
\ifdefined\pdfimageresolution
    \pdfimageresolution= \numexpr \dimexpr1in\relax/\sphinxpxdimen\relax
\fi
%% let collapsible pdf bookmarks panel have high depth per default
\PassOptionsToPackage{bookmarksdepth=5}{hyperref}
%% turn off hyperref patch of \index as sphinx.xdy xindy module takes care of
%% suitable \hyperpage mark-up, working around hyperref-xindy incompatibility
\PassOptionsToPackage{hyperindex=false}{hyperref}
%% memoir class requires extra handling
\makeatletter\@ifclassloaded{memoir}
{\ifdefined\memhyperindexfalse\memhyperindexfalse\fi}{}\makeatother

\PassOptionsToPackage{warn}{textcomp}

\catcode`^^^^00a0\active\protected\def^^^^00a0{\leavevmode\nobreak\ }
\usepackage{cmap}
\usepackage{fontspec}
\defaultfontfeatures[\rmfamily,\sffamily,\ttfamily]{}
\usepackage{amsmath,amssymb,amstext}
\usepackage{polyglossia}
\setmainlanguage{italian}



\setmainfont{FreeSerif}[
  Extension      = .otf,
  UprightFont    = *,
  ItalicFont     = *Italic,
  BoldFont       = *Bold,
  BoldItalicFont = *BoldItalic
]
\setsansfont{FreeSans}[
  Extension      = .otf,
  UprightFont    = *,
  ItalicFont     = *Oblique,
  BoldFont       = *Bold,
  BoldItalicFont = *BoldOblique,
]
\setmonofont{FreeMono}[
  Extension      = .otf,
  UprightFont    = *,
  ItalicFont     = *Oblique,
  BoldFont       = *Bold,
  BoldItalicFont = *BoldOblique,
]



\usepackage[Sonny]{fncychap}
\ChNameVar{\Large\normalfont\sffamily}
\ChTitleVar{\Large\normalfont\sffamily}
\usepackage[,numfigreset=1,mathnumfig]{sphinx}

\fvset{fontsize=\small}
\usepackage{geometry}


% Include hyperref last.
\usepackage{hyperref}
% Fix anchor placement for figures with captions.
\usepackage{hypcap}% it must be loaded after hyperref.
% Set up styles of URL: it should be placed after hyperref.
\urlstyle{same}


\usepackage{sphinxmessages}



        % Start of preamble defined in sphinx-jupyterbook-latex %
         \usepackage[Latin,Greek]{ucharclasses}
        \usepackage{unicode-math}
        % fixing title of the toc
        \addto\captionsenglish{\renewcommand{\contentsname}{Contents}}
        \hypersetup{
            pdfencoding=auto,
            psdextra
        }
        % End of preamble defined in sphinx-jupyterbook-latex %
        

\title{basics book template}
\date{14 dic 2024}
\release{}
\author{basics}
\newcommand{\sphinxlogo}{\vbox{}}
\renewcommand{\releasename}{}
\makeindex
\begin{document}

\pagestyle{empty}
\sphinxmaketitle
\pagestyle{plain}
\sphinxtableofcontents
\pagestyle{normal}
\phantomsection\label{\detokenize{intro::doc}}


\sphinxAtStartPar
If you want ot start a new basics\sphinxhyphen{}book, it could be a good idea to start from this template.

\sphinxAtStartPar
Please check out the Github repo of the project, \sphinxhref{https://github.com/Basics2022}{basics\sphinxhyphen{}book project}, and the \sphinxhref{https://basics2022.github.io/bbooks}{landing page of the project}.
\begin{itemize}
\item {} 
\sphinxAtStartPar
{\hyperref[\detokenize{ch/relativity-special::doc}]{\sphinxcrossref{Special Relativity}}}

\item {} 
\sphinxAtStartPar
{\hyperref[\detokenize{ch/relativity-general::doc}]{\sphinxcrossref{General Relativity}}}

\item {} 
\sphinxAtStartPar
{\hyperref[\detokenize{ch/statistical-mechanics::doc}]{\sphinxcrossref{Statistical Physics}}}

\item {} 
\sphinxAtStartPar
{\hyperref[\detokenize{ch/quantum-mechanics::doc}]{\sphinxcrossref{Quantum Mechanics}}}

\end{itemize}

\sphinxstepscope


\chapter{Special Relativity}
\label{\detokenize{ch/relativity-special:special-relativity}}\label{\detokenize{ch/relativity-special:relativity-special}}\label{\detokenize{ch/relativity-special::doc}}
\sphinxstepscope


\chapter{General Relativity}
\label{\detokenize{ch/relativity-general:general-relativity}}\label{\detokenize{ch/relativity-general:relativity-general}}\label{\detokenize{ch/relativity-general::doc}}
\sphinxstepscope


\chapter{Statistical Physics}
\label{\detokenize{ch/statistical-mechanics:statistical-physics}}\label{\detokenize{ch/statistical-mechanics:statistical-mechanics}}\label{\detokenize{ch/statistical-mechanics::doc}}
\sphinxstepscope


\chapter{Quantum Mechanics}
\label{\detokenize{ch/quantum-mechanics:quantum-mechanics}}\label{\detokenize{ch/quantum-mechanics:id1}}\label{\detokenize{ch/quantum-mechanics::doc}}

\section{Postulates of Quantum Mechanics}
\label{\detokenize{ch/quantum-mechanics:postulates-of-quantum-mechanics}}\begin{itemize}
\item {} 
\sphinxAtStartPar
…

\item {} 
\sphinxAtStartPar
Canonical Commutation Relation (CCR) \sphinxstyleemphasis{and Canonical Anti\sphinxhyphen{}Commutation Realtion…}

\end{itemize}


\section{Non\sphinxhyphen{}relativistic Mechanics}
\label{\detokenize{ch/quantum-mechanics:non-relativistic-mechanics}}

\subsection{Statistical Interpretation}
\label{\detokenize{ch/quantum-mechanics:statistical-interpretation}}

\subsection{Schrodinger Equation}
\label{\detokenize{ch/quantum-mechanics:schrodinger-equation}}\begin{equation*}
\begin{split}i \hbar \dfrac{d}{dt} | \Psi \rangle = \hat{H} | \Psi \rangle \end{split}
\end{equation*}

\subsection{Matrix Mechanics}
\label{\detokenize{ch/quantum-mechanics:matrix-mechanics}}

\section{Many\sphinxhyphen{}body problem}
\label{\detokenize{ch/quantum-mechanics:many-body-problem}}
\sphinxAtStartPar
Wave function with symmetries: Fermions and Bosons







\renewcommand{\indexname}{Indice}
\printindex
\end{document}