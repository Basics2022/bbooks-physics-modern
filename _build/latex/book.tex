%% Generated by Sphinx.
\def\sphinxdocclass{jupyterBook}
\documentclass[letterpaper,10pt,english]{jupyterBook}
\ifdefined\pdfpxdimen
   \let\sphinxpxdimen\pdfpxdimen\else\newdimen\sphinxpxdimen
\fi \sphinxpxdimen=.75bp\relax
\ifdefined\pdfimageresolution
    \pdfimageresolution= \numexpr \dimexpr1in\relax/\sphinxpxdimen\relax
\fi
%% let collapsible pdf bookmarks panel have high depth per default
\PassOptionsToPackage{bookmarksdepth=5}{hyperref}
%% turn off hyperref patch of \index as sphinx.xdy xindy module takes care of
%% suitable \hyperpage mark-up, working around hyperref-xindy incompatibility
\PassOptionsToPackage{hyperindex=false}{hyperref}
%% memoir class requires extra handling
\makeatletter\@ifclassloaded{memoir}
{\ifdefined\memhyperindexfalse\memhyperindexfalse\fi}{}\makeatother

\PassOptionsToPackage{warn}{textcomp}

\catcode`^^^^00a0\active\protected\def^^^^00a0{\leavevmode\nobreak\ }
\usepackage{cmap}
\usepackage{fontspec}
\defaultfontfeatures[\rmfamily,\sffamily,\ttfamily]{}
\usepackage{amsmath,amssymb,amstext}
\usepackage{polyglossia}
\setmainlanguage{english}



\setmainfont{FreeSerif}[
  Extension      = .otf,
  UprightFont    = *,
  ItalicFont     = *Italic,
  BoldFont       = *Bold,
  BoldItalicFont = *BoldItalic
]
\setsansfont{FreeSans}[
  Extension      = .otf,
  UprightFont    = *,
  ItalicFont     = *Oblique,
  BoldFont       = *Bold,
  BoldItalicFont = *BoldOblique,
]
\setmonofont{FreeMono}[
  Extension      = .otf,
  UprightFont    = *,
  ItalicFont     = *Oblique,
  BoldFont       = *Bold,
  BoldItalicFont = *BoldOblique,
]



\usepackage[Bjarne]{fncychap}
\usepackage[,numfigreset=1,mathnumfig]{sphinx}

\fvset{fontsize=\small}
\usepackage{geometry}


% Include hyperref last.
\usepackage{hyperref}
% Fix anchor placement for figures with captions.
\usepackage{hypcap}% it must be loaded after hyperref.
% Set up styles of URL: it should be placed after hyperref.
\urlstyle{same}


\usepackage{sphinxmessages}



        % Start of preamble defined in sphinx-jupyterbook-latex %
         \usepackage[Latin,Greek]{ucharclasses}
        \usepackage{unicode-math}
        % fixing title of the toc
        \addto\captionsenglish{\renewcommand{\contentsname}{Contents}}
        \hypersetup{
            pdfencoding=auto,
            psdextra
        }
        % End of preamble defined in sphinx-jupyterbook-latex %
        

\title{Modern Physics}
\date{Dec 16, 2024}
\release{}
\author{basics}
\newcommand{\sphinxlogo}{\vbox{}}
\renewcommand{\releasename}{}
\makeindex
\begin{document}

\pagestyle{empty}
\sphinxmaketitle
\pagestyle{plain}
\sphinxtableofcontents
\pagestyle{normal}
\phantomsection\label{\detokenize{intro::doc}}


\sphinxAtStartPar
If you want ot start a new basics\sphinxhyphen{}book, it could be a good idea to start from this template.

\sphinxAtStartPar
Please check out the Github repo of the project, \sphinxhref{https://github.com/Basics2022}{basics\sphinxhyphen{}book project}, and the \sphinxhref{https://basics2022.github.io/bbooks}{landing page of the project}.
\begin{itemize}
\item {} 
\sphinxAtStartPar
{\hyperref[\detokenize{ch/relativity-special::doc}]{\sphinxcrossref{Special Relativity}}}

\item {} 
\sphinxAtStartPar
{\hyperref[\detokenize{ch/relativity-general::doc}]{\sphinxcrossref{General Relativity}}}

\item {} 
\sphinxAtStartPar
{\hyperref[\detokenize{ch/statistical-mechanics::doc}]{\sphinxcrossref{Statistical Physics}}}

\item {} 
\sphinxAtStartPar
{\hyperref[\detokenize{ch/quantum-mechanics::doc}]{\sphinxcrossref{Quantum Mechanics}}}

\end{itemize}

\sphinxstepscope


\chapter{Special Relativity}
\label{\detokenize{ch/relativity-special:special-relativity}}\label{\detokenize{ch/relativity-special:relativity-special}}\label{\detokenize{ch/relativity-special::doc}}
\sphinxstepscope


\chapter{General Relativity}
\label{\detokenize{ch/relativity-general:general-relativity}}\label{\detokenize{ch/relativity-general:relativity-general}}\label{\detokenize{ch/relativity-general::doc}}
\sphinxstepscope


\chapter{Statistical Physics}
\label{\detokenize{ch/statistical-mechanics:statistical-physics}}\label{\detokenize{ch/statistical-mechanics:statistical-mechanics}}\label{\detokenize{ch/statistical-mechanics::doc}}
\sphinxstepscope


\chapter{Quantum Mechanics}
\label{\detokenize{ch/quantum-mechanics:quantum-mechanics}}\label{\detokenize{ch/quantum-mechanics:id1}}\label{\detokenize{ch/quantum-mechanics::doc}}

\section{Mathematical tools for quantum mechanics}
\label{\detokenize{ch/quantum-mechanics:mathematical-tools-for-quantum-mechanics}}\label{ch/quantum-mechanics:Operator}
\begin{sphinxadmonition}{note}{Definition 1 (Operator)}


\end{sphinxadmonition}
\label{ch/quantum-mechanics:Adjoint Operator}
\begin{sphinxadmonition}{note}{Definition 2 (Adjoint operator)}



\sphinxAtStartPar
Given an operator \(\hat{A}: U \rightarrow V\), its self\sphinxhyphen{}adjoint \(\hat{A}^*: V \rightarrow U\) is the operator s.t.
\begin{equation*}
\begin{split}(\mathbf{v}, \ \hat{A} \mathbf{u})_{V} = (\mathbf{u}, \hat{A}^* \mathbf{v} )_{U}\end{split}
\end{equation*}
\sphinxAtStartPar
holds for \(\forall \mathbf{u} \in U, \ \mathbf{v} \in V\).
\end{sphinxadmonition}
\label{ch/quantum-mechanics:Self-Adjoint Operator}
\begin{sphinxadmonition}{note}{Definition 3 (Hermitian (self\sphinxhyphen{}adjoint) operator)}



\sphinxAtStartPar
If \(\hat{A}: U \rightarrow U\), it is a self\sphinxhyphen{}adjoint operator if \(\hat{A}^* = \hat{A}\).
\end{sphinxadmonition}

\sphinxAtStartPar
Self\sphinxhyphen{}adjoint operators have real eigenvalues, and orthogonal eigenvectors (at least those associated to different eigenvalues; those associated with the same eigenvalues can be used to build an orthogonal set of vectors with orthogonalization process).


\section{Postulates of Quantum Mechanics}
\label{\detokenize{ch/quantum-mechanics:postulates-of-quantum-mechanics}}\begin{itemize}
\item {} 
\sphinxAtStartPar
…

\item {} 
\sphinxAtStartPar
Canonical Commutation Relation (CCR) \sphinxstyleemphasis{and Canonical Anti\sphinxhyphen{}Commutation Relation…}

\end{itemize}


\section{Non\sphinxhyphen{}relativistic Mechanics}
\label{\detokenize{ch/quantum-mechanics:non-relativistic-mechanics}}

\subsection{Statistical Interpretation}
\label{\detokenize{ch/quantum-mechanics:statistical-interpretation}}

\subsubsection{Wave function}
\label{\detokenize{ch/quantum-mechanics:wave-function}}
\sphinxAtStartPar
The state of a system is described by a wave function \(|\Psi\rangle\)

\sphinxAtStartPar
\sphinxstylestrong{todo}
\begin{itemize}
\item {} 
\sphinxAtStartPar
properties: domain, image,…

\item {} 
\sphinxAtStartPar
unitary \(1 = \langle \Psi | \Psi \rangle = \left| \Psi \right|^2\), for statistical interpretation of \(\left| \Psi \right|^2\) as a density probability function

\end{itemize}


\subsubsection{Operators and Observables}
\label{\detokenize{ch/quantum-mechanics:operators-and-observables}}
\sphinxAtStartPar
Physical \sphinxstylestrong{observable} quantities are represented by {\hyperref[\detokenize{ch/quantum-mechanics:quantum-mechanics-math-operators-self-adjoint}]{\sphinxcrossref{\DUrole{std,std-ref}{Hermitian operators}}}}. Possible outcomes of measurement are the eigenvalues of the operator

\sphinxAtStartPar
Given \(\hat{A}\) and the set of its eigenvectors \(\{ |A_i \rangle \}_i\) (\sphinxstylestrong{todo} \sphinxstyleemphasis{continuous or discrete spectrum…, need to treat this difference quite in details}), with associated eigenvalues \(\{ a_i \}_i\)
\begin{equation*}
\begin{split}\hat{A} |A_i \rangle = a_i |A_i\rangle\end{split}
\end{equation*}\begin{equation*}
\begin{split}| \Psi \rangle = | A_i \rangle \langle A_i | \Psi \rangle =  | A_i \rangle \Psi_i^A \end{split}
\end{equation*}\begin{equation*}
\begin{split}\langle A_j | \Psi \rangle = \langle A_j | A_i \rangle \langle A_i | \Psi \rangle = \Psi_j^A \end{split}
\end{equation*}
\sphinxAtStartPar
and thus
\begin{equation*}
\begin{split}\begin{aligned}
  \Psi_j^A    & = \langle A_j  | \Psi \rangle \\
  \Psi_j^{A*} & = \langle \Psi | A_j \rangle \\
\end{aligned}\end{split}
\end{equation*}\begin{itemize}
\item {} 
\sphinxAtStartPar
identity operator \(\sum_i | A_i \rangle \langle A_i | = \mathbb{I}\), since
\begin{equation*}
\begin{split}\sum_i | A_i \rangle \langle A_i | \Psi \rangle = \sum_i | A_i \rangle \langle A_i | \Psi_j^A A_j \rangle =
  \sum_i | A_i \rangle \delta_{ij} \Psi_j^A = \sum_i | A_i \rangle  \Psi_i^A  = | \Psi \rangle\end{split}
\end{equation*}
\item {} 
\sphinxAtStartPar
Normalization:
\begin{equation*}
\begin{split}1 =\langle \Psi | \Psi \rangle = \Psi_j^{A*} \underbrace{\langle A_j | A_i \rangle}_{\delta_{ij}} \Psi_i^{A} = \sum_i \left| \Psi_i^A \right|^2\end{split}
\end{equation*}
\sphinxAtStartPar
with \(|\Psi_i^A|^2\) that can be interpreted as the probability of finding the system in state \(|\Psi_i^a\rangle\)

\item {} 
\sphinxAtStartPar
Expected value of the physical quantity in the a state \(|\Psi\rangle\), with possible values \(a_i\) with probability \(|\Psi_i^A|^2\)
\begin{equation*}
\begin{split}\begin{aligned}
    \bar{A}_{\Psi} & = \sum_i a_i |\Psi_i^A|^2 = \\
            & = \sum_i a_i \Psi_i^{A*} \Psi_i^A =  \\ 
            & = \sum_i a_i \langle \Psi | A_i \rangle \langle A_i | \Psi \rangle = \\
            & = \langle \Psi | \left( \sum_i a_i | A_i \rangle \langle A_i | \right) | \Psi \rangle = \\
            & = \langle \Psi | \hat{A} | \Psi \rangle = \\
  \end{aligned}\end{split}
\end{equation*}
\sphinxAtStartPar
since an operator \(\hat{A}\) can be written as a function of its eigenvalues and eigenvectors
\begin{equation*}
\begin{split}\begin{aligned}
   \left( \sum_i a_i | A_i \rangle \langle A_i | \right) \Psi \rangle 
   & = \left( \sum_i a_i | A_i \rangle \langle A_i | \right) c_k | A_k \rangle = \\
   & =  \sum_i a_i | A_i \rangle c_i = \\
   & =  \sum_i \hat{A} | A_i \rangle c_i = \\
   & = \hat{A} \sum_i | A_i \rangle c_i = \hat{A} | \Psi \rangle \ . 
   \end{aligned}\end{split}
\end{equation*}
\end{itemize}




\subsubsection{Space Representation}
\label{\detokenize{ch/quantum-mechanics:space-representation}}
\sphinxAtStartPar
\sphinxstylestrong{Position operator} \(\hat{\mathbf{r}}\) has eigenvalues \(\mathbf{r}\) identifying the possible measurements of the position
\begin{equation*}
\begin{split}\hat{\mathbf{r}} | \mathbf{r} \rangle = \mathbf{r} | \mathbf{r} \rangle \ ,\end{split}
\end{equation*}
\sphinxAtStartPar
being \(\mathbf{r}\) the result of the measurement (position in space, mathematically it could be a vector), and \(| \mathbf{r} \rangle\) the state function corresponding to the measurement \(\mathbf{r}\) of the position.
\begin{itemize}
\item {} 
\sphinxAtStartPar
Result of measurement, \(\mathbf{r}\), is a position in space. As an example, it could be a point in an Euclidean space \(P \in E^n\). It could be written using properties of Dirac’s delta “function”
\begin{equation*}
\begin{split}
    \mathbf{r} = \int_{\mathbf{r}'} \delta (\mathbf{r}'-\mathbf{r}) \, \mathbf{r}' d \mathbf{r}'
  \end{split}
\end{equation*}
\item {} 
\sphinxAtStartPar
Projection of wave function over eigenstates of position operator
\begin{equation*}
\begin{split}\begin{aligned}
  \langle \mathbf{r} | \Psi \rangle(t) = \Psi(\mathbf{r}, t)
  & = \int_{\mathbf{r}'} \delta(\mathbf{r} - \mathbf{r}') \Psi(\mathbf{r}',t) d \mathbf{r}' = \\
  & = \int_{\mathbf{r}'} \langle \mathbf{r} | \mathbf{r}' \rangle \Psi(\mathbf{r}',t) d \mathbf{r}' = \\
  & = \int_{\mathbf{r}'} \langle \mathbf{r} | \mathbf{r}' \rangle \langle \mathbf{r}' | \Psi \rangle (t) d \mathbf{r}' = \\
  & = \langle \mathbf{r} | \underbrace{\left( \int_{\mathbf{r}'} | \mathbf{r}' \rangle \langle \mathbf{r}' | d \mathbf{r}' \right)}_{= \hat{\mathbf{1}}} | \Psi \rangle (t) \ .
  \end{aligned}\end{split}
\end{equation*}
\item {} 
\sphinxAtStartPar
having used orthogonality (\sphinxstylestrong{todo} \sphinxstyleemphasis{why? provide definition and examples of operators with continuous spectrum})
\begin{equation*}
\begin{split}\langle \mathbf{r}' | \mathbf{r} \rangle = \delta(\mathbf{r}' - \mathbf{r})\end{split}
\end{equation*}
\item {} 
\sphinxAtStartPar
Expansion of a state function \(|\Psi\rangle (t)\) over the basis of the position operator
\begin{equation*}
\begin{split}\begin{aligned}
  | \Psi \rangle (t) = \hat{\mathbf{1}} | \Psi \rangle(t) 
    = \left( \int_{\mathbf{r}'} | \mathbf{r}' \rangle \langle \mathbf{r}' d \mathbf{r}' \right) | \Psi \rangle(t) 
    = \int_{\mathbf{r}'} | \mathbf{r}' \rangle \langle \mathbf{r}' | \Psi \rangle(t) \, d \mathbf{r}' \ .
  \end{aligned}\end{split}
\end{equation*}
\item {} 
\sphinxAtStartPar
Unitariety and probability density
\begin{equation*}
\begin{split}\begin{aligned}
  1 = \langle \Psi | \Psi \rangle (t) 
    & = \langle \Psi | \left( \int_{\mathbf{r}'} | \mathbf{r}' \rangle \langle \mathbf{r}' d \mathbf{r}' \right) | \Psi \rangle \\
    & = \int_{\mathbf{r}'} \langle \Psi | \mathbf{r}' \rangle \langle \mathbf{r}' | \Psi \rangle \, d \mathbf{r}' \\
    & = \int_{\mathbf{r}'} \Psi^*(\mathbf{r}',t) \Psi(\mathbf{r}',t) \, d \mathbf{r}' \\
    & = \int_{\mathbf{r}'} \left| \Psi(\mathbf{r}',t) \right|^2 \, d \mathbf{r}' \\
  \end{aligned}\end{split}
\end{equation*}
\sphinxAtStartPar
and thus \(\left|\Psi(\mathbf{r},t)\right|^2\) can be interpreted as the \sphinxstylestrong{probability density function} of measuring position of the system equal to \(\mathbf{r}'\).

\item {} 
\sphinxAtStartPar
Average value of the operator
\begin{equation*}
\begin{split}\begin{aligned}
  \bar{\mathbf{r}} & = \langle \Psi | \hat{\mathbf{r}} | \Psi \rangle = \\
  & = \int_{\mathbf{r}'} \langle \Psi | \mathbf{r}' \rangle \langle \mathbf{r}' | d \mathbf{r}' \ | \hat{\mathbf{r}} | \ \int_{\mathbf{r}''} | \mathbf{r}'' \rangle \langle \mathbf{r}'' | \Psi \rangle \, d \mathbf{r}'' \\
  & = \int_{\mathbf{r}'} \int_{\mathbf{r}''} \langle \Psi | \mathbf{r}' \rangle \langle \mathbf{r}' | \hat{\mathbf{r}} |  \mathbf{r}'' \rangle \langle \mathbf{r}'' | \Psi \rangle \, d \mathbf{r}'  d \mathbf{r}'' = \\
  & = \int_{\mathbf{r}'} \int_{\mathbf{r}''} \langle \Psi | \mathbf{r}' \rangle \underbrace{\langle \mathbf{r}' |  \mathbf{r}'' \rangle}_{=\delta(\mathbf{r}'-\mathbf{r}'')} \mathbf{r}'' \langle \mathbf{r}'' | \Psi \rangle \, d \mathbf{r}'  d \mathbf{r}'' = \\
  & = \int_{\mathbf{r}'} \langle \Psi | \mathbf{r}' \rangle \mathbf{r}' \langle \mathbf{r}' | \Psi \rangle \, d \mathbf{r}' = \\
  & = \int_{\mathbf{r}'} \Psi^*(\mathbf{r}',t) \ \mathbf{r}' \ \Psi(\mathbf{r}',t) \, d \mathbf{r}' = \\
  & = \int_{\mathbf{r}'}  \left| \Psi(\mathbf{r}',t) \right|^2 \, \mathbf{r}' \, d \mathbf{r}' \ .
  \end{aligned}\end{split}
\end{equation*}
\end{itemize}


\subsubsection{Momentum Representation}
\label{\detokenize{ch/quantum-mechanics:momentum-representation}}
\sphinxAtStartPar
\sphinxstylestrong{Momentum operator} as the limit of…\sphinxstylestrong{todo} \sphinxstyleemphasis{prove the expression of the momentum operator as the limit of the generator of translation}
\begin{equation*}
\begin{split}\langle \mathbf{r} | \hat{\mathbf{p}} = - i \hbar \nabla \langle \mathbf{r} | \end{split}
\end{equation*}\begin{itemize}
\item {} 
\sphinxAtStartPar
Spectrum
\begin{equation*}
\begin{split}\hat{\mathbf{p}} | \mathbf{p} \rangle = \mathbf{p} | \mathbf{p} \rangle\end{split}
\end{equation*}\begin{equation*}
\begin{split}\langle \mathbf{r} | \hat{\mathbf{p}} | \mathbf{p} \rangle = - i \hbar \nabla \langle \mathbf{r} | \mathbf{p} \rangle = \mathbf{p} \langle \mathbf{r} | \mathbf{p} \rangle\end{split}
\end{equation*}
\sphinxAtStartPar
and thus the eigenvectors in space base \(\mathbf{p}(\mathbf{r}) = \langle \mathbf{r} | \mathbf{p} \rangle\) are the solution of the differential equation
\begin{equation*}
\begin{split}- i \hbar \nabla \mathbf{p}(\mathbf{r}) = \mathbf{p} \mathbf{p}(\mathbf{r}) \ ,\end{split}
\end{equation*}
\sphinxAtStartPar
that in Cartesian coordinates reads
\begin{equation*}
\begin{split}- i \hbar \partial_j p_k(\mathbf{r}) = p_j p_k(\mathbf{r})\end{split}
\end{equation*}\begin{equation*}
\begin{split}p_k(\mathbf{r}) = p_{k,0} \exp \left[ i \frac{p_j}{\hbar} r_j \right]\end{split}
\end{equation*}
\sphinxAtStartPar
or
\begin{equation*}
\begin{split}\langle \mathbf{r} | \mathbf{p} \rangle = \mathbf{p}(\mathbf{r}) = \mathbf{p}_0 \exp \left[ i \frac{\mathbf{p} \cdot \mathbf{r}}{\hbar} \right]\end{split}
\end{equation*}
\sphinxAtStartPar
\sphinxstylestrong{todo}
\begin{itemize}
\item {} 
\sphinxAtStartPar
normalization factor \(\frac{1}{(2 \pi)^{\frac{3}{2}}}\)
\begin{equation*}
\begin{split}\mathscr{F}\{ \delta(x) \}(k) = \int_{-\infty}^{\infty} \delta(x) e^{-ikx} \, dx = 1\end{split}
\end{equation*}
\item {} 
\sphinxAtStartPar
Fourier transform and inverse Fourier transform: definitions and proofs (link to a math section)

\item {} 
\sphinxAtStartPar
representation in basis of wave vector operator \(\hat{\mathbf{k}}\), \(\hat{\mathbf{p}} = \hbar \hat{\mathbf{k}}\)

\end{itemize}

\end{itemize}


\subsubsection{From position to momentum representation}
\label{\detokenize{ch/quantum-mechanics:from-position-to-momentum-representation}}
\sphinxAtStartPar
Momentum and wave vector, \(\mathbf{p} = \hbar \mathbf{k}\)
\begin{equation*}
\begin{split}\begin{aligned}
  \langle \mathbf{p} | \Psi \rangle
  & = \langle \mathbf{p} | \int_{\mathbf{r}'} | \mathbf{r}' \rangle \langle \mathbf{r}' | \Psi\rangle d \mathbf{r}' = \\
  & =   \int_{\mathbf{r}'} \langle \mathbf{p} | \mathbf{r}' \rangle \langle \mathbf{r}' | \Psi\rangle d \mathbf{r}' = \\
  & = \frac{1}{(2\pi)^{3/2}} \int_{\mathbf{r}'} \exp \left[ i \frac{\mathbf{p} \cdot \mathbf{r}}{\hbar} \right] \langle \mathbf{r}' | \Psi\rangle d \mathbf{r}' = \\
\end{aligned}\end{split}
\end{equation*}
\sphinxAtStartPar
Relation between position and wave\sphinxhyphen{}number representation can be represented with a Fourier transform
\begin{equation*}
\begin{split}\begin{aligned}
  \langle \mathbf{k} | \Psi \rangle
  & = \langle \mathbf{k} | \int_{\mathbf{r}'} | \mathbf{r}' \rangle \langle \mathbf{r}' | \Psi\rangle d \mathbf{r}' = \\
  & =   \int_{\mathbf{r}'} \langle \mathbf{k} | \mathbf{r}' \rangle \langle \mathbf{r}' | \Psi\rangle d \mathbf{r}' = \\
  & = \frac{1}{(2\pi)^{3/2}} \int_{\mathbf{r}'} \exp \left[ i \mathbf{k} \cdot \mathbf{r}' \right] \langle \mathbf{r}' | \Psi\rangle d \mathbf{r}' = \\
  & = \frac{1}{(2\pi)^{3/2}} \int_{\mathbf{r}'} \exp \left[ i \mathbf{k} \cdot \mathbf{r}' \right] \Psi(\mathbf{r}') d \mathbf{r}' = \\
  & = \mathscr{F}\{ \Psi(\mathbf{r}) \} (\mathbf{k})
\end{aligned}\end{split}
\end{equation*}

\subsection{Schrodinger Equation}
\label{\detokenize{ch/quantum-mechanics:schrodinger-equation}}\begin{equation*}
\begin{split}i \hbar \dfrac{d}{dt} | \Psi \rangle = \hat{H} | \Psi \rangle \end{split}
\end{equation*}
\sphinxAtStartPar
being \(\hat{H}\) the Hamiltonian operator and \(|\Psi\rangle\) the wave function, as a function of time \(t\) as an independent variable.


\subsubsection{Stationary States}
\label{\detokenize{ch/quantum-mechanics:stationary-states}}
\sphinxAtStartPar
Eigenspace of the Hamiltonian operator
\begin{equation*}
\begin{split}\hat{H} |\Psi_k \rangle = E_k |\Psi_k \rangle \ ,\end{split}
\end{equation*}
\sphinxAtStartPar
with \(E_k\) possible values of energy measurements. \sphinxstyleemphasis{If no eigenstates with the same eigenvalue exists, then…otherwise…}
\sphinxstyleemphasis{Without external influence} \sphinxstylestrong{todo} \sphinxstyleemphasis{be more detailed!}, energy values and eigenstates of the systems are constant in time.

\sphinxAtStartPar
Thus, exapnding the state of the system \(|\Psi\rangle\) over the stationary states gives \(|\Psi_k\rangle\), \(|\Psi\rangle = |\Psi_k\rangle c_k(t)\), and inserting in Schrodinger equation
\begin{equation*}
\begin{split}i \hbar \dot{c}_k |\Psi_k\rangle = c_k  E_k |\Psi_k\rangle\end{split}
\end{equation*}
\sphinxAtStartPar
and exploiting orthogonality of eigenstates, a diagonal system for the amplitudes of stationary states ariese,
\begin{equation*}
\begin{split}i \hbar \dot{c}_k = c_k E_k \ .\end{split}
\end{equation*}
\sphinxAtStartPar
whose solution reads
\begin{equation*}
\begin{split}c_k(t) = c_{k,0} \exp \left[- i \frac{E_k}{\hbar} t \right]\end{split}
\end{equation*}
\sphinxAtStartPar
Thus the state of the system evolves like a superposition of monochromatic waves with frequencies \(\omega_k = \frac{E_k}{\hbar}\),
\begin{equation*}
\begin{split}| \Psi \rangle = | \Psi_k \rangle c_k(t) = | \Psi_k \rangle c_{k,0}\exp\left[-i \frac{E_k}{\hbar} t \right] \ .\end{split}
\end{equation*}\begin{equation*}
\begin{split}\begin{aligned}
 \dfrac{d}{dt} \bar{A}
 & = \dfrac{d}{dt} \left( \langle \Psi | \hat{A} | \Psi \rangle \right) = \\
 & = \dfrac{d}{dt} \langle \Psi | \hat{A} | \Psi \rangle + \langle \Psi | \frac{d \hat{A}}{dt} | \Psi \rangle + \langle \Psi | \hat{A} \frac{d}{dt} | \Psi \rangle = \\
 & = \langle \Psi | \frac{d \hat{A}}{dt} | \Psi \rangle + \frac{i}{\hbar} \langle \Psi | \hat{H} \hat{A} | \Psi \rangle - \frac{i}{\hbar} \langle \Psi | \hat{A} \hat{H} | \Psi \rangle = \\
 & = \langle \Psi | \left( \frac{i}{\hbar} [ \hat{H}, \hat{A} ] + \frac{d \hat{A}}{dt} \right) | \Psi \rangle \ .
\end{aligned}\end{split}
\end{equation*}

\subsubsection{Pictures}
\label{\detokenize{ch/quantum-mechanics:pictures}}\begin{itemize}
\item {} 
\sphinxAtStartPar
Schrodinger

\item {} 
\sphinxAtStartPar
Heisenberg

\item {} 
\sphinxAtStartPar
Interaction

\end{itemize}


\paragraph{Schrodinger}
\label{\detokenize{ch/quantum-mechanics:schrodinger}}
\sphinxAtStartPar
If \(\hat{H}\) not function of time
\begin{equation*}
\begin{split}| \Psi \rangle (t) = \exp\left[ - i \frac{\hat{H}}{\hbar} (t-t_0) \right] | \Psi \rangle(t_0) = \hat{U}(t,t_0) | \Psi \rangle(t_0) \end{split}
\end{equation*}\begin{equation*}
\begin{split}\bar{A} = \langle \Psi | \hat{A} | \Psi \rangle = \langle \Psi_0 | \hat{U}^*(t,t_0) \hat{A} \hat{U}(t,t_0) | \Psi_0 \rangle\end{split}
\end{equation*}

\paragraph{Heisenberg}
\label{\detokenize{ch/quantum-mechanics:heisenberg}}
\sphinxAtStartPar
…

\sphinxAtStartPar
for \(\hat{H}\) independent from time \(t\),
\begin{equation*}
\begin{split}\begin{aligned}
  \dfrac{d}{dt} \bar{\mathbf{r}} & = \overline{\frac{i}{\hbar} \left[ \hat{H}, \hat{\mathbf{r}} \right]} \\
  \dfrac{d}{dt} \bar{\mathbf{p}} & = \overline{\frac{i}{\hbar} \left[ \hat{H}, \hat{\mathbf{p}} \right]} \\
\end{aligned}\end{split}
\end{equation*}\subsubsection*{Hamiltonian Mechanics}

\sphinxAtStartPar
From Lagrange equations
\begin{equation*}
\begin{split}\dfrac{d}{dt} \left( \frac{\partial L}{\partial \dot{q}} \right) - \frac{\partial L}{\partial q} = Q_q\end{split}
\end{equation*}
\sphinxAtStartPar
\(q\) generalized coordinates, \(p:= \frac{\partial L}{\partial \dot{q}}\) generalized momenta.

\sphinxAtStartPar
Hamiltonian
\begin{equation*}
\begin{split}H(p,q,t) = p \dot{q} - L(\dot{q}, q, t)\end{split}
\end{equation*}
\sphinxAtStartPar
Increment of the Hamiltonian
\begin{equation*}
\begin{split}dH = \partial_p H dp + \partial_q H dq + \partial_t H dt \end{split}
\end{equation*}\begin{equation*}
\begin{split}\begin{aligned}
  dH & = \dot{q} dp + p d \dot{q} - \partial_{\dot{q}} L d \dot{q} - \partial_{q} L d q - \partial_t L d t = \\
     & = \dot{q} dp - \partial_{q} L d q - \partial_t L d t = \\
     & = \dot{q} dp - \left( \dot{p} + Q_q \right) d q - \partial_t L d t = \\
\end{aligned}\end{split}
\end{equation*}\begin{equation*}
\begin{split}\begin{cases}
  \frac{\partial H}{\partial p} = \dot{q} \\
  \frac{\partial H}{\partial q} = - \frac{\partial L}{\partial q} = - \dot{p} + Q_q \\
  \frac{\partial H}{\partial t} = - \frac{\partial L}{\partial t}
\end{cases}\end{split}
\end{equation*}
\sphinxAtStartPar
Physical quantity \(f(p(t), q(t), t)\). Its time derivative reads
\begin{equation*}
\begin{split}\begin{aligned}
\frac{d f}{dt}
  & = \frac{\partial f}{\partial p} \dot{p} + \frac{\partial f}{\partial q} \dot{q} + \frac{\partial f}{\partial t} = \\
  & = \frac{\partial f}{\partial p} \left[ - \frac{\partial H}{\partial q} + Q_q \right] + \frac{\partial f}{\partial q} \frac{\partial H}{\partial p} + \frac{\partial f}{\partial t} = \\
  & = \{ H, f \} + \partial_t f + Q_q \partial_p f
\end{aligned}\end{split}
\end{equation*}
\sphinxAtStartPar
If \(Q_q = 0\), the correspondence between quantum mechanics and classical mechanics
\begin{equation*}
\begin{split}\frac{d f}{d t} = \{ H, f \} + \partial_t f \qquad \leftrightarrow \qquad \dfrac{d}{dt} \overline{\hat{f}} = \overline{\frac{i}{\hbar} [ \hat{H}, \hat{f} ]} + \overline{\frac{\partial \hat{f}}{\partial t}}\end{split}
\end{equation*}\begin{equation*}
\begin{split}\{ H, f \} \qquad \leftrightarrow \qquad \frac{i}{\hbar}[\hat{H}, \hat{f}]\end{split}
\end{equation*}

\paragraph{Interaction}
\label{\detokenize{ch/quantum-mechanics:interaction}}

\subsection{Matrix Mechanics}
\label{\detokenize{ch/quantum-mechanics:matrix-mechanics}}

\subsubsection{Attualization of 1925 papers}
\label{\detokenize{ch/quantum-mechanics:attualization-of-1925-papers}}


\sphinxAtStartPar
…to find the canonical commutation relation,
\begin{equation*}
\begin{split}[ \hat{\mathbf{r}}, \hat{\mathbf{p}} ] = i \hbar  \mathbb{I} \, \hat{\mathbf{1}} \ .\end{split}
\end{equation*}\begin{equation*}
\begin{split}\begin{aligned}
\left[ \hat{\mathbf{r}}, \hat{\mathbf{p}} \right]
 & = \hat{\mathbf{r}} \hat{\mathbf{p}} - \hat{\mathbf{p}} \hat{\mathbf{r}} = \\
 & = \hat{\mathbf{r}} \int_{\mathbf{r}} | \mathbf{r} \rangle \langle \mathbf{r} | d \mathbf{r} \hat{\mathbf{p}}
   - \hat{\mathbf{p}} \int_{\mathbf{r}} | \mathbf{r} \rangle \langle \mathbf{r} | d \mathbf{r} \ \hat{\mathbf{r}} \ \int_{\mathbf{r}'} | \mathbf{r}' \rangle \langle \mathbf{r}' | d \mathbf{r}' = \\
 & = - \int_{\mathbf{r}} \mathbf{r} | \mathbf{r} \rangle i \hbar \nabla \langle \mathbf{r} | \, d \mathbf{r} - \hat{\mathbf{p}} \int_{\mathbf{r}} \int_{\mathbf{r}'} | \mathbf{r} \rangle \mathbf{r}' \underbrace{\langle \mathbf{r} |\mathbf{r}' \rangle}_{\delta(\mathbf{r}-\mathbf{r}')} \langle \mathbf{r}' | d \mathbf{r}' = \\
 & = - \int_{\mathbf{r}} \mathbf{r} | \mathbf{r} \rangle i \hbar \nabla \langle \mathbf{r} | \, d \mathbf{r} - \hat{\mathbf{p}} \int_{\mathbf{r}} \mathbf{r} | \mathbf{r} \rangle \langle \mathbf{r} | d \mathbf{r} = \\
 & = - \int_{\mathbf{r}} \mathbf{r} | \mathbf{r} \rangle i \hbar \nabla \langle \mathbf{r} | \, d \mathbf{r} - \int_{\mathbf{r}'} | \mathbf{r} \rangle \langle \mathbf{r} | d \mathbf{r} \ \hat{\mathbf{p}} \ \int_{\mathbf{r}'} \mathbf{r}' | \mathbf{r}' \rangle \langle \mathbf{r}' | d \mathbf{r}' = \\
 & = - \int_{\mathbf{r}} \mathbf{r} | \mathbf{r} \rangle i \hbar \nabla \langle \mathbf{r} | \, d \mathbf{r} + \int_{\mathbf{r}} | \mathbf{r} \rangle i \hbar \nabla \langle \mathbf{r} | d \mathbf{r} \int_{\mathbf{r}'} \mathbf{r}' | \mathbf{r}' \rangle \langle \mathbf{r}' | d \mathbf{r}' = \dots
\end{aligned}\end{split}
\end{equation*}\begin{equation*}
\begin{split}\begin{aligned}
\left[ \hat{\mathbf{r}}, \hat{\mathbf{p}} \right] | \Psi \rangle 
 & = - \int_{\mathbf{r}} \mathbf{r} | \mathbf{r} \rangle i \hbar \nabla \Psi(\mathbf{r},t) + \int_{\mathbf{r}} | \mathbf{r} \rangle i \hbar \nabla \left( \mathbf{r} \Psi(\mathbf{r},t) \right) = \\
 & = - \int_{\mathbf{r}} | \mathbf{r} \rangle i \hbar \left[ \mathbf{r} \nabla \Psi(\mathbf{r},t) + \mathbb{I} \Psi(\mathbf{r},t) + \mathbf{r} \nabla \Psi(\mathbf{r},t) \right] = \\
 & = i \hbar \underbrace{\int_{\mathbf{r}} | \mathbf{r} \rangle \langle \mathbf{r} | d \mathbf{r}}_{\hat{\mathbf{1}}} \, | \Psi \rangle \ ,
\end{aligned}\end{split}
\end{equation*}
\sphinxAtStartPar
and since \(| \Psi \rangle\) is arbitrary
\begin{equation*}
\begin{split}\left[ \hat{\mathbf{r}}, \hat{\mathbf{p}} \right] = i \hbar \mathbb{I} \hat{\mathbf{1}} \ . \end{split}
\end{equation*}\begin{equation*}
\begin{split}\left[ \hat{r}_a, \hat{p}_b \right] = i \hbar \delta_{ab} \ . \end{split}
\end{equation*}

\subsection{Heisenberg Uncertainty “principle”}
\label{\detokenize{ch/quantum-mechanics:heisenberg-uncertainty-principle}}\begin{itemize}
\item {} 
\sphinxAtStartPar
Heisenberg uncertainty “principle” is a relation between product of variance of two physical quantities and their commutation,

\item {} 
\sphinxAtStartPar
\sphinxstylestrong{todo} \sphinxstyleemphasis{relation with measurement process and outcomes. Commutation as a measurement process: first measure \(B\) and then \(A\), or first measure \(A\) and then \(B\)}

\end{itemize}
\begin{equation*}
\begin{split}\sigma_A \sigma_B \ge \frac{1}{2} \left|\overline{[\hat{A}, \hat{B}]}\right| \ .\end{split}
\end{equation*}\subsubsection*{Proof of Heisenberg uncertainty “principle”}
\begin{equation*}
\begin{split}\begin{aligned}
 \sigma^2_A \sigma^2_B
 & = \langle \Psi | \left(\hat{A} - \bar{A} \right)^2 | \Psi \rangle\langle \Psi | \left(\hat{B} - \bar{B} \right)^2 | \Psi \rangle = \\
 & = \langle (\hat{A} - \bar{A} ) \Psi |  (\hat{A} - \bar{A} ) \Psi \rangle \langle  (\hat{B} - \bar{B} ) \Psi |  (\hat{B} - \bar{B} ) \Psi \rangle = \\
 & = \| ( \hat{A} - \bar{A} ) \Psi \|^2 \| ( \hat{B} - \bar{B} ) \Psi \|^2 = \\
 & \ge \left| \langle ( \hat{A} - \bar{A} ) \Psi | ( \hat{B} - \bar{B} ) \Psi  \rangle \right|^2 = \\
 & = \left| \langle \Psi | (\hat{A} - \bar{A})(\hat{B} - \bar{B}) \Psi \rangle \right|^2 = \\
 & = \left| \langle \Psi | \hat{A}\hat{B} - \hat{A}\bar{B} - \bar{A}\hat{B} + \bar{A}\bar{B} | \Psi \rangle \right|^2 = \\
 & = \left| \langle \Psi | \hat{A}\hat{B} - \bar{A}\bar{B} | \Psi \rangle \right|^2 = \\
 & = \left| \frac{\langle \Psi | \hat{A}\hat{B} - \hat{B}\hat{A} | \Psi \rangle}{2i} \right|^2 = \\
 & = \frac{ \left|\langle \Psi | [\hat{A}, \hat{B}] | \Psi \rangle \right|^2}{4} = \frac{1}{4} \left| \overline{[\hat{A}, \hat{B}]} \right|^2
\end{aligned}\end{split}
\end{equation*}
\sphinxAtStartPar
having used Cauchy triangle inequality and
\begin{equation*}
\begin{split}|z| = \frac{\text{re}\{z\} + \text{re}\{z^*\}}{2} = \frac{\text{im}\{z\} - \text{im}\{z^*\}}{2i}\end{split}
\end{equation*}
\sphinxAtStartPar
Hesienberg uncertainty principles applied to position and momentum reads
\begin{equation*}
\begin{split}\sigma_{r_a} \sigma_{p_b} \ge \frac{1}{2} \left|\overline{[\hat{r}_a, \hat{p}_b]}\right| = \frac{\hbar}{2}  \delta_{ab} \ .\end{split}
\end{equation*}

\section{Many\sphinxhyphen{}body problem}
\label{\detokenize{ch/quantum-mechanics:many-body-problem}}
\sphinxAtStartPar
Wave function with symmetries: Fermions and Bosons






\renewcommand{\indexname}{Proof Index}
\begin{sphinxtheindex}
\let\bigletter\sphinxstyleindexlettergroup
\bigletter{Adjoint Operator}
\item\relax\sphinxstyleindexentry{Adjoint Operator}\sphinxstyleindexextra{ch/quantum\sphinxhyphen{}mechanics}\sphinxstyleindexpageref{ch/quantum-mechanics:\detokenize{Adjoint Operator}}
\indexspace
\bigletter{Operator}
\item\relax\sphinxstyleindexentry{Operator}\sphinxstyleindexextra{ch/quantum\sphinxhyphen{}mechanics}\sphinxstyleindexpageref{ch/quantum-mechanics:\detokenize{Operator}}
\indexspace
\bigletter{Self\sphinxhyphen{}Adjoint Operator}
\item\relax\sphinxstyleindexentry{Self\sphinxhyphen{}Adjoint Operator}\sphinxstyleindexextra{ch/quantum\sphinxhyphen{}mechanics}\sphinxstyleindexpageref{ch/quantum-mechanics:\detokenize{Self-Adjoint Operator}}
\end{sphinxtheindex}

\renewcommand{\indexname}{Index}
\printindex
\end{document}